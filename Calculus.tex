\documentclass{report}
\usepackage{cmap}
\usepackage[utf8]{inputenc}
\usepackage[russian]{babel}
\usepackage{setspace,amsmath}
\usepackage{amssymb}
\usepackage{amsthm}
\usepackage{amsfonts}
\usepackage{scalerel}
\usepackage{graphicx}
\usepackage{float}
\usepackage{wrapfig}
\usepackage{pgfplots}
\usepackage[unicode, pdftex]{hyperref}
\usepackage{mdsymbol}

\pgfplotsset{compat=1.9}

\def\stretchint#1{\vcenter{\hbox{\stretchto[440]{\displaystyle\int}{#1}}}}
\def\scaleint#1{\vcenter{\hbox{\scaleto[3ex]{\displaystyle\int}{#1}}}}

\theoremstyle{definition}
\newtheorem{definition}{Определение}[section]
\newtheorem{example}{Пример}
\newtheorem*{effect}{Следствие}
\newtheorem{statement}{Утверждение}[section]
\newtheorem*{remark}{Замечание}
\newtheorem{lemma}{Лемма}[section]
\newtheorem{theorem}{Теорема}[section]

\title{Математический Анализ \\ 3 семестр}
\author{Данил Заблоцкий}
\date{\today}

\begin{document}

\maketitle
\tableofcontents

\chapter{Название}

\section{Название}

\begin{effect}
  $D$ - область в $\mathbb{R}^n, \ f:D\rightarrow\mathbb{R}$ дифференцируема на $D$ и $\forall x \in D \ df(x) = 0$, то есть $\forall i \ \frac{\delta f}{\delta x_i} = 0$. Тогда $f - const$.
\end{effect}

\begin{proof}
  $x_0 \in D, \ B(x_0, \rho)\subset D, \ \forall x \in B(x_0, \rho), \ [x_0,x] \subset B(x_0,\rho) \subset D. \quad f(x) - f(x_0) = f'(\xi)(x-x_0). \quad f(x) - f(x_0) = 0 \implies f(x) = f(x_0)$.

  Построим путь из точки $x_0$ к некоторой точке $x \in D, \quad \gamma:[0;1]\rightarrow D, \quad \gamma(0) = x_0, \ \gamma(1) = x$. По определению пути, $\gamma$ - непрерывна. Тогда $\exists\delta:$
  \begin{equation*}
    \forall 0 \leqslant t \leqslant \delta \implies \forall x \in B(x_0,\rho), \quad \gamma(t) \in B(x_0,\rho) \implies f(\gamma(t)) = f(x_0), \ t \in [0,\delta]
  \end{equation*}

  Пусть $\Delta = \sup\delta \implies f(\gamma(\Delta)) = f(x_0)$. Покажем, что $\Delta = 1$. Пусть $\Delta < 1 (0 + 1)$. Построим шар $B(\gamma(\Delta),\rho_\Delta) \subset D$. Тогда $\exists \epsilon > 0:$
  \begin{equation*}
    \Delta - \epsilon < t < \Delta + \epsilon
  \end{equation*}

  Но тогда $f(\gamma(\Delta + \epsilon)) = f(x_0)$ (так как точка $\gamma(\Delta + \epsilon) \in B(\gamma(\Delta), \rho_\Delta)$). Это противоречит с тем, что $\Delta = \sup \delta \implies \Delta = 1 \implies \gamma(1) = x, \ f(x) = f(x_0) \implies$ так как $x \in D$ - произвольная точка, то имеем, что $\forall x \in D:$
  \begin{equation*}
    f(x) = f(x_0) \implies f(x) - const
  \end{equation*}
\end{proof}

\begin{theorem}[Достаточное условие дифференцируемости функции]
  Пусть $D$ - область в $\mathbb{R}^n, \ f:D\rightarrow\mathbb{R}, \ f$ имеет непрерывную часть произведения в каждой окрестности точки $x \in D$.

  Тогда $f$ - дифференцируема в точке $x$.
\end{theorem}

\begin{proof}
  Без ограничения общности, что окрестность точки $x_0 \in D$ является шаром $B(x_0,\rho)\subset D$.

  Пусть $h:x_0 + h \in B(x_0,\rho)$. Здесь $x_0 = (x^1, \ x^2, \ \ldots, \ x^n), \quad x_0 + h = (x^1 + h^1, \ x^2 + h^2, \ \ldots, \ x^n + h^n)$. Заметим, что точки $x_1 = (x^1, \ x^2 + h^2, \ \ldots, \ x^n + h^n), \ x^2 = (x^1, \ x^2, \ \ldots, \ x^n + h^n), \ \ldots, \ x_{n-1} = (x^1, \ x^2, \ \ldots, x^{n-1} \ x^n + h^n) \in B(x_0,\rho)$.

  $f(x_0 + h) - f(x_0) = f(x_0 + h) - f(x_1) + f(x_1) - f(x_2) + f(x_2) - \ldots - f(x_{n-1}) + f(x_{n-1}) - f(x_0) = f(x^1 + h^1, \ \ldots, \ x^n + h^n) - f(x^1, \ x^2 + h^2, \ \ldots, \ x^n + h^n) + f(x^1, \ x^2 + h^2, \ \ldots, \ x^n + h^n) - f(x^1, \ x^2, \ \ldots, \ x^n + h^n) + f(x^1, \ x^2, \ \ldots, \ x^n + h^n) - \ldots - f(x^1, \ x^2, \ \ldots, \ x^{n-1}, \ x^n) + f(x^1, \ x^2, \ \ldots, \ x^{n-1}, \ x^n + h^n) - f(x^1, \ x^2, \ \ldots, \ x^n) = \left| Lagranj \ theorem \ for \ 1 \ variable \right| = \frac{\delta f}{\delta x_1}(x^1 + \theta^1 h^1, \ x^2 + h^2, \ \ldots, \ x^n + h^n) \cdot h^1 + \frac{\delta f}{\delta x^2}(x^1, \ x^2 + \theta^2 h^2, \ \ldots, \ x^n + h^n) \cdot h^2 + \ldots + \frac{\delta f}{\delta x^n}(x^1, \ x^2, \ \ldots, \ x^n + \theta^n h^n) \cdot h^n$.

  Используя непрерывность частных производных, запишем: $f(x_0 + h) - f(x_0) = \frac{\delta f}{\delta x_1}(x^1, \ x^2, \ \ldots, \ x^n) \cdot h^1 + \alpha^1(h^1) + \ldots + \frac{\delta f}{\delta x_n}(x^1, \ x^2, \ \ldots, \ x^n) \cdot h^2 + \alpha^n(h^n)$, где $\alpha^1,\alpha^2,\ldots,\alpha^n$ стремятся к нулю при $h\rightarrow0$.

  Это означает, что $f(x_0 + h) - f(x_0) = L(x_0)\cdot h + \underset{h\rightarrow0}{o}(h)$, где $L(x_0) = \frac{\delta f}{\delta x_1}(x_0)h^1 + \ldots + \frac{\delta f}{\delta x^n}(x_0) \cdot h^n = df(x_0) \implies$ по определению $f(x)$ дифференцируема в точке $x_0$.
\end{proof}

\section{Производные высших порядков}

\begin{definition}[Вторая производная по двум переменным]
  Пусть $f:D\rightarrow\mathbb{R}, \ D$ - область в $\mathbb{R}^n$. Производная по переменной $x^i$ от производной по переменной $x^j$ называется \textbf{второй производной} функции $f$ по переменным $x^i, x^j$ и обозначается:
  \begin{equation*}
    \frac{\delta^2 f}{\delta x^i \delta x^j}(x), \quad f_{x^ix^j}''(x)
  \end{equation*}
\end{definition}

\begin{theorem}[О смешанных производных]
  Пусть $D$ - область в $\mathbb{R}^n, \ f: D\rightarrow\mathbb{R}, \ x \in D, \ f$ имеет в $D$ непрерывно смешанные производные (2-го порядка).

  Тогда эти производные не зависят от порядка дифференцирования.
\end{theorem}

\begin{proof}
  Пусть $\frac{\delta^2 f}{\delta x^i \delta x^j}$ и $\frac{\delta^2 f}{\delta x^j \delta x^i}$ - непрерывны в точке $x \in D$. Так как остальные переменные фиксированы, то можно считать, что $f$ зависит только от двух переменных.

  Тогда $D\subset\mathbb{R}^2, \ f:D\rightarrow\mathbb{R}$ и $\frac{\delta^2 f}{\delta x \delta y}$ и $\frac{\delta^2 f}{\delta y \delta x}$ - непрерывны в точке $x_0 = (x,y) \in D$.

  Покажем, что $\frac{\delta^2 f}{\delta x \delta y} = \frac{\delta^2 f}{\delta y \delta x}$.

  Рассмотрим функции $\phi(t) = f(x+t\cdot\Delta x, \ y + \Delta y) - f(x+t\cdot\Delta x, \ y), \ \psi(t) = f(x + \Delta x, \ y + t\cdot\Delta y) - f(x, \ y + t \cdot \Delta y), \ t \in [0;1]$.

  Имеем, что $\phi(1) - \phi(0) = f(x + \Delta x, \ y + \Delta y) = f(x + \Delta x, \ y) - f(x, \ y + \Delta y) + f(x, \ y)$.

  $\psi(1) - \psi(0) = f(x + \Delta x, \ y + \Delta y) - f(x, \ y + \Delta y) - f(x + \Delta x, \ y) + f(x, \ y)$.

  Тогда $\phi(1) - \phi(0) = \psi(1) - \psi(0)$.

  Тут нужно дописать, фотки в галерее
\end{proof}

\begin{proof}
  Рассмотрим функцию $\phi(t) = f(x + th)$. Применим формулу Тейлора к $\phi(t)$:
  \begin{center}
    $\phi(1) = \phi(0) + \frac{1}{1!} \cdot \phi'(0) \cdot (1-0) + \frac{1}{2!} \cdot \phi''(0) \cdot (1-0)^2 + \frac{1}{3!} \cdot \phi'''(0) \cdot (1-0)^3 + \ldots$ \\
    $ + \frac{1}{k!} \cdot \phi^{(k)} \cdot (1-0)^k$ \\
  \end{center}
  $\phi(1) = f(x + h); \quad \phi(0) = f(x);$ \\

  $\phi'(0) = f'(x + th) \cdot (x + th)_k'\big|_{t = 0} = \left(\frac{\delta f(x + th)}{\delta x^1} \ \frac{\delta f(x + th)}{\delta x^2} \ \ldots \ \frac{\delta f(x + th)}{\delta x^n}\right) \cdot \left(\begin{array}{c}
        h^1    \\
        h^2    \\
        \vdots \\
        h^n
      \end{array}\right) =$ \\
  $= (\frac{\delta f(x + th)}{\delta x^1} \cdot h' + \frac{\delta f(x+th)}{\delta x^2} \cdot h^2 + \ldots + \frac{\delta f}{\delta x^n}(x + th)\cdot h^n) \big|_{t=0} = \frac{\delta f(x)}{\delta x^1}\cdot h^1 + \frac{\delta f}{\delta x^2}(x)\cdot h^2 + \ldots + \frac{\delta f}{\delta x^n}(x) \cdot h^n = (\frac{\delta}{\delta x^1} \cdot h^1 + \ldots + \frac{\delta}{\delta x^n}\cdot h^n)f(x)$. \\

  $\phi''(0) = (\sum_{i = 1}^{n} \frac{\delta f(x + th)}{\delta x^i}h^i)_t' \big|_{t = 0} = (\sum_{i = 1}^{n} \sum_{j = 1}^{n} \frac{\delta^2 f(x + th)}{\delta x^i \ \delta x^j} h^i h^j) \big|_{t = 0} = \sum_{i = 1}^{n}\sum_{j = 1}^{n}\frac{\delta^2 f(x)}{\delta x^i \ \delta x^j} h^i h^j = (\frac{\delta}{\delta x^1} + \ldots + \frac{\delta}{\delta x^n}h^n)^2 f(x)$

  И так далее. Подстановки получившиеся выражаем в $(*)$ и получим искомое.
\end{proof}

\section{Экстремумы функций многих переменных}

\begin{definition}
  Пусть $X$ - метрическое пространство $f:X \rightarrow \mathbb{R}$. Точка $x_0 \in X$ называется \textbf{точкой локального максимума (минимума)}, если:
  \begin{equation*}
    \exists u(x_0)\subset X: \ \forall x \in u(x_0) \quad f(x)\leqslant f(x_0) \ (f(x)\geqslant f(x_0))
  \end{equation*}

  Точки локального максимума и минимума называются \textbf{точками локального экстремума}.
\end{definition}

\begin{theorem}[Необходимое условие локального экстремума]
  Пусть $D$ - область в $\mathbb{R}^n, \ f : D\rightarrow\mathbb{R}, \ x_0 \in D$ - точка локального экстремума, тогда в точке $x_0$
  \begin{equation*}
    \forall i = \overline{1,n} \quad \frac{\delta f(x_0)}{\delta x^i} = 0
  \end{equation*}
\end{theorem}

\begin{proof}
  Фиксируем все переменные за исключением $x^i$, тогда можно рассмотреть $f(x^1, \ldots, x^i, \ldots x^n)$ как функцию одной переменной, для которой $x_0$ - точка локального экстремума $\implies \frac{\delta f}{\delta x^i}(x_0) = 0$.

  $i$ - произвольная $\implies \forall i$ - выполняется.
\end{proof}

\begin{definition}
  Пусть $D$ - область в $\mathbb{R}^n, \ f:D\rightarrow\mathbb{R}^k$ - дифференцируемо в точке $x_0 \in D, \ x_0$ называется \textbf{критической точкой функции} $f(x)$, если
  \begin{equation*}
    rank \mathfrak{I} f(x_0) < min(n,k),
  \end{equation*}
  где $\mathfrak{I} f(x_0)$ - матрица Якоби функции $f(x_0)$.
\end{definition}

\begin{example}
  $f:\mathbb{R}^3 \rightarrow\mathbb{R}^2$ \\

  $f(x,y,z) = \left(\begin{array}{c}
        x\cdot y \\
        y - z
      \end{array}\right) = \left(\begin{array}{c}
        u \\
        v
      \end{array}\right)$ \\

  $\mathfrak{I} f(x,y,z) = \left( \begin{array}{ccc}
        \frac{\delta u}{\delta x} & \frac{\delta u}{\delta y} & \frac{\delta u}{\delta z} \\
        \frac{\delta v}{\delta x} & \frac{\delta v}{\delta y} & \frac{\delta v}{\delta z}
      \end{array} \right) = \left( \begin{array}{ccc}
        y & x & 0  \\
        0 & 1 & -1
      \end{array} \right) \implies \left\{ \begin{array}{l}
      x = 0 \\
      y = 0 \\
      z = t
    \end{array}\right. - (x_0)$ (критические точки)

  $n = 3, \ k = 2$ \\

  Множество точек прямой, получаемой пересечением плоскостей $x = 0$ и $y = 0$ - множество критических точек функции $f(x,y,z)$.
\end{example}

\begin{definition}
  Пусть $D$ - область в $\mathbb{R}^n, \ f:D\rightarrow\mathbb{R}^n, \ f$ имеет непрерывные вторые производные в точке $x_0 \in D$. На касательном пространстве $T\mathbb{R}^n_{(x_0)}$ определим квадратичную форму
  \begin{equation*}
    Q(h) = \sum_{i,j = 1}^{n}\frac{\delta^2 f}{\delta x^i \delta x^j}(x_0)h^i\cdot h^j
  \end{equation*}
  \begin{equation*}
    Q:T\mathbb{R}^n \rightarrow \mathbb{R}
  \end{equation*}
\end{definition}

\begin{center}
  ${\Large 02.10}$
\end{center}

\begin{example}
  $S = \mathbb{R}^n$ - поверхность в $\mathbb{R}^n$
  \begin{equation*}
    t^i(x^i) = \frac{\pi}{2}\cdot \arctan x^i
  \end{equation*}
  $\mathfrak{I} = \left(\begin{array}{cccc}
        \frac{\delta t^1}{\delta x^1} & \frac{\delta t^1}{\delta x^2} & \ldots & \frac{\delta t^1}{\delta x^n} \\
        \frac{\delta t^2}{\delta x^1} & \frac{\delta t^2}{\delta x^2} & \ldots & \frac{\delta t^2}{\delta x^n} \\
        \vdots                        & \vdots                        & \ddots & \vdots                        \\
        \frac{\delta t^n}{\delta x^1} & \frac{\delta t^n}{\delta x^2} & \ldots & \frac{\delta t^n}{\delta x^n} \\
      \end{array}\right) = \left(\begin{array}{cccc}
        \frac{2}{\pi} \cdot \frac{1}{1 + (x^1)^2} & 0                                         & \ldots & 0                                         \\
        0                                         & \frac{2}{\pi} \cdot \frac{1}{1 + (x^2)^2} & \ldots & 0                                         \\
        \vdots                                    & \vdots                                    & \ddots & \vdots                                    \\
        0                                         & 0                                         & \ldots & \frac{2}{\pi} \cdot \frac{1}{1 + (x^n)^2}
      \end{array}\right)$
\end{example}

\begin{statement}
  Пусть задана система уравнений
  \begin{equation*}
    \left\{\begin{array}{l}
      F^1(x^1,\ldots,x^n) = 0 \\
      \vdots                  \\
      F^{n-k}(x^1,\ldots,x^n) = 0
    \end{array}\right.,
  \end{equation*}
  здесь $F^i(x)\in C^{(1)}$.

  Кроме того,
  \begin{equation*}
    \left|\begin{array}{ccc}
      \frac{\delta F^1}{\delta x^1}     & \ldots & \frac{\delta F^1}{\delta x^n}     \\
      \vdots                            & \ddots & \vdots                            \\
      \frac{\delta F^{n-k}}{\delta x^1} & \ldots & \frac{\delta F^{n-k}}{\delta x^n}
    \end{array}\right|(x) \ne 0, \quad \forall x \in \mathbb{R}^n
  \end{equation*}

  Тогда решение этой системы является $k$-мерной поверхностью в $\mathbb{R}^n$.
\end{statement}

\begin{proof}
  По теореме о неявной функции, система
  \begin{equation*}
    \left\{\begin{array}{l}
      F^1(x^1,\ldots,x^n) = 0 \\
      \vdots                  \\
      F^{n-k}(x^1,\ldots,x^n) = 0
    \end{array}\right.
  \end{equation*}
  эквивалентна системе
  \begin{equation*}
    \left\{\begin{array}{l}
      x^{k+1} = f^1(x^1,\ldots,x^k) \\
      x^{k+2} = f^2(x^1,\ldots,x^k) \\
      \vdots                        \\
      x^{n-k} = f^{n-k}(x^1,\ldots,x^k)
    \end{array}\right.
  \end{equation*}

  Получим: \\

  $t^1 = x^1 \\
    t^2 = x^2 \\
    \vdots \\
    t^k = x^k \\
    t^{k+1} = x^{k+1} - f^1(x^1,\ldots,x^k) = 0 \\
    t^{k+2} = x^{k+2} - f^2(x^1,\ldots,x^k) = 0 \\
    \vdots \\
    t^{n} = x^{n-k} - f^{n-k}(x^1,\ldots,x^k) = 0$ \\

  Таким образом построенное отображение является диффиоморфизмом $\implies$ решение системы
  \begin{center}
    $\left\{\begin{array}{l}
        F^1(x^1,\ldots,x^n) = 0 \\
        \vdots                  \\
        F^{n-k}(x^1,\ldots,x^n) = 0
      \end{array}\right.$ -- $k$-мерная поверхность в $\mathbb{R}^n$.
  \end{center}
\end{proof}

\begin{definition}[Локальная карта или параметризация поверхности, касательное пространство]
  Пусть $S$ -- $k$-мерная поверхность в $\mathbb{R}^n, \ x_0 \in S$ и $\phi: U(x_0) \rightarrow I^n$ - диффиоморфизм:
  \begin{equation*}
    \phi(U(x_0)\cap S) = I^k
  \end{equation*}

  Ограничение $\phi^{-1}$ на $I^k$ будем называть \textbf{локальной картой} или \textbf{параметризацией поверхности} $S$ в окрестности точки $x_0$.

  \textbf{Касательным пространством} (или плоскостью) к $S$ в точке $x_0$ называется $k$-мерная плоскость, заданная уравнением
  \begin{equation*}
    x = x_0 + x'(0) \cdot t, \quad x_0 = (x^1_0,x^1_0,\ldots,x^n_0)
  \end{equation*}
  \begin{equation*}
    x(t) = \left\{\begin{array}{l}
      x^1(t^1,\ldots,t^k) \\
      x^2(t^1,\ldots,t^k) \\
      \vdots              \\
      x^n(t^1,\ldots,t^k)
    \end{array}\right., \quad x'(t) = \left(\begin{array}{ccc}
        \frac{\delta x^1}{\delta t^1} & \ldots & \frac{\delta x^1}{\delta t^k} \\
        \vdots                        & \ddots & \vdots                        \\
        \frac{\delta x^n}{\delta t^1} & \ldots & \frac{\delta x^n}{\delta t^k} \\
      \end{array}\right) (t)
  \end{equation*}

  Таким образом касательное пространство задается системой (из $x = x_0 + x'(0) \cdot t$)
  \begin{equation*}
    \left\{\begin{array}{l}
      x^1 = x^1_0 + \frac{\delta x^1}{\delta t^1} (0)t^1 + \ldots + \frac{\delta x^1}{t^k}(0)t^k \\
      x^2 = x^2_0 + \frac{\delta x^2}{\delta t^1} (0)t^1 + \ldots + \frac{\delta x^2}{t^k}(0)t^k \\
      \vdots                                                                                     \\
      x^n = x^n_0 + \frac{\delta x^n}{\delta t^1} (0)t^1 + \ldots + \frac{\delta x^n}{t^k}(0)t^k
    \end{array}\right.
  \end{equation*}
\end{definition}

\begin{example}
  \begin{enumerate}
    \item Пусть $\gamma = \gamma (t)$ - гладкая кривая в $\mathbb{R}^3, \ \gamma:\left\{\begin{array}{c}
              x = x(t) \\
              y = y(t) \\
              z = z(t)
            \end{array}\right.$.

          Обозначим $x_0 = x(0), \ y_0 = y(0), \ z_0 = z(0)$.

          $x = x_0 + x'(0) \cdot t: \ x = x_0 + x'(0) \cdot t$ - касательное пространство к кривой $\gamma$ в точке $x_0$
          \begin{center}
            $\left\{\begin{array}{l}
                x = x_0 + x'(0) \cdot t \\
                y = y_0 + y'(0) \cdot t \\
                z = z_0 + z'(0) \cdot t \\
              \end{array}\right.$, иначе $\left\{\begin{array}{l}
                x - x_0 = x'(0)\cdot t \\
                y - y_0 = y'(0)\cdot t \\
                z - z_0 = z'(0)\cdot t \\
              \end{array}\right.$
          \end{center}
          \begin{equation*}
            \frac{x - x_0}{x'(0)} = \frac{y - y_0}{y'(0)} = \frac{z - z_0}{z'(0)} = t
          \end{equation*}

    \item $x^2 + y^2 + z^2 = 1$

          Пусть $z_0 > 0$, тогда в окрестности точки $(x_0,y_0,z_0)$ сферу можно параметризовать следующими уравнениями:
          \begin{equation*}
            \left\{\begin{array}{l}
              x = u \\
              y = v \\
              z = \sqrt{1 - u^2 - v^2}
            \end{array}\right.
          \end{equation*}

          Касательное пространство к сфере в точке $(x_0,y_0,z_0)$:
          \begin{equation*}
            \left(\begin{array}{c}
              x \\
              y \\
              z
            \end{array}\right) = \left(\begin{array}{c}
              x_0 \\
              y_0 \\
              z_0
            \end{array}\right) + \left(\begin{array}{cc}
              \frac{\delta x}{\delta u} & \frac{\delta x}{\delta v} \\
              \frac{\delta y}{\delta u} & \frac{\delta y}{\delta v} \\
              \frac{\delta z}{\delta u} & \frac{\delta z}{\delta v} \\
            \end{array}\right) (u_0,v_0) \cdot \left(\begin{array}{c}
              u \\
              v
            \end{array}\right) \implies
          \end{equation*}
          \begin{equation*}
            \left\{\begin{array}{l}
              x = x_0 + \frac{\delta x}{\delta u}(u_0,v_0) \cdot u + \frac{\delta x}{\delta v}(u_0,v_0)\cdot v \\
              y = y_0 + \frac{\delta y}{\delta u}(u_0,v_0) \cdot u + \frac{\delta y}{\delta v}(u_0,v_0)\cdot v \\
              z = z_0 + \frac{\delta z}{\delta u}(u_0,v_0) \cdot u + \frac{\delta z}{\delta v}(u_0,v_0)\cdot v \\
            \end{array}\right.
          \end{equation*}

          \begin{equation*}
            \left\{\begin{array}{l}
              x = x_0 + u \\
              y = y_0 + v \\
              z = z_0 - \frac{u_0}{\sqrt{1 - u_0^2 - v_0^2}}\cdot u - \frac{v_0}{\sqrt{1 - u_0^2 - v_0^2}}\cdot v
            \end{array}\right.
          \end{equation*}
  \end{enumerate}
\end{example}

\begin{statement}
  Пусть $S$ -- $k$-мерная поверхность в $\mathbb{R}^n$ задается системой уравнений
  \begin{center}
    $\left\{\begin{array}{l}
        F^1(x^1,\ldots,x^n) = 0 \\
        \vdots                  \\
        F^{n-k}(x^1,\ldots,x^n) = 0,
      \end{array}\right.$, причем $\left|\begin{array}{ccc}
        \frac{\delta F^1}{\delta x^{k+1}}     & \ldots & \frac{\delta F^1}{\delta x^n}     \\
        \vdots                                & \ddots & \vdots                            \\
        \frac{\delta F^{n-1}}{\delta x^{k+1}} & \ldots & \frac{\delta F^{n-1}}{\delta x^n} \\
      \end{array}\right|(x_0) \ne 0$.
  \end{center}

  Тогда касательная плоскость к $S$ в точке $x_0$ задается системой уравнений:
  \begin{equation*}
    \left\{\begin{array}{l}
      \frac{\delta F^1}{\delta x^1}(x_0)\cdot (x^1-x^1_0) + \frac{\delta F^1}{\delta x^2}(x_0)(x^2-x_0^2) + \ldots + \frac{\delta F^1}{\delta x^n}(x_0)(x^n - x_0^n) = 0 \\
      \vdots                                                                                                                                                             \\
      \frac{\delta F^{n-k}}{\delta x^1}(x_0)\cdot (x^1-x^1_0) + \frac{\delta F^{n-k}}{\delta x^2}(x_0)(x^2-x_0^2) + \ldots + \frac{\delta F^{n-k}}{\delta x^n}(x_0)(x^n - x_0^n) = 0
    \end{array}\right.
  \end{equation*}
  или кратко
  \begin{equation*}
    F'(x_0)\cdot(x-x_0) = 0
  \end{equation*}
\end{statement}

\begin{proof}
  Обозначим $(x^1,\ldots,x^k) = u, \ (x^{k+1},\ldots,x^n) = v$
  \begin{equation*}
    F = \left(\begin{array}{c}
      F^1    \\
      \vdots \\
      f^{n-k}
    \end{array}\right)
  \end{equation*}

  Тогда условия утверждения запишем в виде:
  \begin{equation*}
    F(u,v) = 0, \quad |F'_v(u_0,v_0)| \ne 0
  \end{equation*}

  Тогда по теореме о неявной функции система $\left\{\begin{array}{l}
      F^1(x^1,\ldots,x^n) = 0 \\
      \vdots                  \\
      F^{n-k}(x^1,\ldots,x^n) = 0,
    \end{array}\right.$ эквивалентна системе
  \begin{equation*}
    \left\{\begin{array}{l}
      u = u \\
      v = f(u)
    \end{array}\right.
  \end{equation*}

  Тогда касательная плоскость задается (роль $t = \left(\begin{array}{c}
        t^1    \\
        \vdots \\
        t^k
      \end{array}\right)$ играет $u = \left(\begin{array}{c}
        x^1    \\
        \vdots \\
        x^k
      \end{array}\right)$).

  Тогда систему можно записать в виде:
  \begin{equation*}
    \left\{\begin{array}{l}
      x^1 = t^1                       \\
      \vdots                          \\
      x^k = t^k                       \\
      x^{k + 1} = f^1(t^1,\ldots,t^k) \\
      \vdots                          \\
      x^n = f^{n-k}(t^1,\ldots,t^k)
    \end{array}\right.
  \end{equation*}
  \begin{equation*}
    t_0 = (t_0^k,\ldots,t_0^k) = (x_0^1,\ldots,x_0^k)
  \end{equation*}

  \begin{equation*}
    x'(t_0) = \left(\begin{array}{ccc}
      \frac{\delta x^1}{\delta t^1}     & \ldots & \frac{\delta x^1}{\delta t^k}     \\
      \vdots                            & \ddots & \vdots                            \\
      \frac{\delta x^k}{\delta t^1}     & \ldots & \frac{\delta x^k}{\delta t^k}     \\
      \frac{\delta f^1}{\delta t^1}     & \ldots & \frac{\delta f^1}{\delta t^k}     \\
      \vdots                            & \ddots & \vdots                            \\
      \frac{\delta f^{n-k}}{\delta t^1} & \ldots & \frac{\delta f^{n-k}}{\delta t^k}
    \end{array}\right)(t_0)
  \end{equation*}
  \begin{equation*}
    x = x_0 + x'(t_0)\cdot t
  \end{equation*}
  $\left\{\begin{array}{l}
      x^1 = x_0^1 + 1 \cdot t^1                                                                                         \\
      \vdots                                                                                                            \\
      x^k = x_0^k + 1 \cdot t^k                                                                                         \\
      x^{k+1} = x^{k+1} + \frac{\delta f^1}{\delta t^k}(t_0) \cdot t^1 + \ldots + \frac{\delta f^1}{\delta t^k}(x_0)t^k \\
      \vdots                                                                                                            \\
      x^n = x^n + \frac{\delta f^{n-k}}{\delta t^k}(t_0) t^1 + \ldots + \frac{\delta f^{n-k}}{\delta t^k}(x_0)t^k
    \end{array}\right., \quad \begin{array}{l}
      u = (x^1,\ldots,x^k)     \\
      v = (x^{k+1},\ldots,x^n) \\
      \left\{\begin{array}{l}
               u = u \\
               v = f(u)
             \end{array}\right.
    \end{array}$ \\

  \begin{equation*}
    f'(u_0) = -[F_x'(u_0,v_0)]^{-1} \cdot F_u' (u_0,v_0)
  \end{equation*}

  $\left\{\begin{array}{l}
      x^1 - x_0^1 = t^1                                                                                                                                 \\
      \vdots                                                                                                                                            \\
      x^k - x_0^k = t^k                                                                                                                                 \\
      x^{k+1} - x^{k+1}_0 = \frac{\delta f^1}{\delta t^k}(t_0) \cdot (x^1 - x_0^1) + \ldots + \frac{\delta f^1}{\delta t^k}(x_0)\cdot (x^k - x_0^k)     \\
      \vdots                                                                                                                                            \\
      x^{n} - x^{n}_0 = \frac{\delta f^{n-k}}{\delta t^k}(t_0) \cdot (x^1 - x_0^1) + \ldots + \frac{\delta f^{n-k}}{\delta t^k}(x_0)\cdot (x^k - x_0^k) \\
    \end{array}\right.$ \\

  $\left\{\begin{array}{l}
      u - u_0 = u - u_0 \\
      v - v_0 = f'(u_0) \cdot (u - u_0)
    \end{array}\right. \implies \left\{\begin{array}{l}
      u - u_0 = u - u_0 \\
      v - v_0 = -[F_v'(u_0,v_0)]^{-1} \cdot F_u'(u_0,v_0)
    \end{array}\right. \implies [F_v'(u_0,v_0)](v-v_0) + F_u'(u_0,v_0)\cdot (u - u_0) = 0$
\end{proof}

Итак, мы вывели, что если поверхность задана линейным уравнением $\left\{\begin{array}{l}
    F'(x^1,\ldots,x^n) = 0 \\
    \vdots                 \\
    F^{n-k}(x^1,\ldots,x^n) = 0
  \end{array}\right.$ или $P(x) = 0, \ F = \left(\begin{array}{c}
      F'(x)  \\
      \vdots \\
      F^{n-k}(x)
    \end{array}\right), \ x = (x^1,\ldots,x^n), \ x_0 = (x_0^1,\ldots,x_0^n)$.

Тогда уравнение касательной плоскости кратко записывается:
\begin{equation*}
  F_x'(x_0)\cdot(x-x_0) = 0
\end{equation*}

Обозначим $x - x_0 = \xi$, то есть:
\begin{equation*}
  \xi = \left(\begin{array}{c}
    \xi^1  \\
    \vdots \\
    \xi^n
  \end{array}\right) = \left(\begin{array}{c}
    x'-x_0' \\
    \vdots  \\
    x^n-x_0^n
  \end{array}\right)
\end{equation*}

Таким образом получаем, что уравнение касательной пространства имеет вид:
\begin{equation*}
  F_x'(x_0)\cdot \xi = 0
\end{equation*}

Таким образом касательнаое пространство к поверхности заданной уравнением $F(x) = 0$ в точке $x_0$ состоит из векторов $\xi$, удовлетворяет уравнению
\begin{equation}
  F_x'(x_0)\cdot \xi = 0
\end{equation}

\begin{theorem}[О структуре касательных пространства]
  Пусть $S$ -- $k$-мерная поверхность в $\mathbb{R}^n, \ x_0 \in S$. Тогда касательное пространство $TS_{x_0}$ в точке $x_0$ состоит из направляющих векторов касательных к гладким кривым на поверхности $S$, проходящих через точку $x_0$.
\end{theorem}

\begin{proof}
  Пусть $x = x(t)$ -- гладкая кривая в $\mathbb{R}^n$, то есть \\
  $\left\{\begin{array}{l}
      x' = x'(t) \\
      \vdots     \\
      x^n = x^n(t)
    \end{array}\right., \ t \in \mathbb{R}, \ x_0 = x(t_0)$

  Касательный вектор в точке $x_0$ к кривой имеет вид
  \begin{equation*}
    \left(\begin{array}{c}
      \frac{dx'}{dt}(t_0) \\
      \vdots              \\
      \frac{dx^n}{dt}(t_0)
    \end{array}\right) = \left(\begin{array}{c}
      x^{1'}(t_0) \\
      \vdots      \\
      x^{1n}(t_0)
    \end{array}\right)
  \end{equation*}
  \begin{enumerate}
    \item Пусть $S$ -- $k$-мерная поверхность, задана системой уравнений $F(x) = 0$ и пусть $x = x(t)$ -- гладкая кривая на $S$. Покажем, что вектор $x'(t_0) = \left(\begin{array}{c}
                \frac{dx'}{dt}(t_0) \\
                \vdots              \\
                \frac{dx^n}{dt}(t_0)
              \end{array}\right), \ x'(t_0) \in TS_{x_0}, \ x_0 = x(t_0)$, то есть покажем, что $x'(t_0)$ удовлетворяет уравнению $F_x'(x_0)\cdot \xi=0$.

          Так как кривая $x = x(t)$ лежит на $S$, то $F(x(t)) = 0$ -- верно. Продифференцируем $F(x(t)) = 0$ по $t$ в точке $x_0$:
          \begin{equation*}
            F_x'(x_0) \cdot x'(t_0) = 0
          \end{equation*}
          -- это и есть уравнение касательного пространства, то есть $x'(t_0)$ удовлетворяет уравнению касательной кривой $F_x'(x_0)\cdot \xi = 0$.

    \item Пусть $\xi = (\xi^1,\xi^2,\ldots,\xi^n) \in TS_{x_0}$, то есть $\xi$ удовлетворяет уравнению $F_x'(x_0)\cdot \xi = 0$
  \end{enumerate}

  Покажем, что $\exists$ гладкая кривая $l$ на поверхности $S$:
  \begin{enumerate}
    \item $x_0 \in l$
    \item $\xi$ ялвяется направляющим вектором касательной к $l$ в точке $x_0$
  \end{enumerate}

  Поверхность $S$ задана системой уравнений:
  \begin{equation}
    \left\{\begin{array}{l}
      F^1(x) = 0 \\
      \vdots     \\
      F^{n-k}(x) = 0
    \end{array}\right.
  \end{equation}

  Пусть
  \begin{equation*}
    \left|\begin{array}{ccc}
      \frac{\delta F'}{\delta x^{k+1}}      & \cdots & \frac{\delta F'}{\delta x^n}      \\
      \vdots                                & \ddots & \vdots                            \\
      \frac{\delta F^{n-k}}{\delta x^{k+1}} & \cdots & \frac{\delta F^{n-k}}{\delta x^n}
    \end{array}\right| (x_0) \ne 0
  \end{equation*}

  По теореме о неявной функции, система $(1.4)$ эквивалентна системе
  \begin{equation}
    \left\{\begin{array}{l}
      x^{k+1} = f^1(x^1,\ldots,x^k) \\
      \vdots                        \\
      x^n = f^{n-k}(x^1,\ldots,x^k)
    \end{array}\right\}
  \end{equation}

  Обозначим $u = (x',\ldots,x^k), \ v = (x^{k+1},\ldots,x^n)$, тогда $(1.3)$ имеет вид
  \begin{equation*}
    v = f(u)
  \end{equation*}

  Тогда по утверждению касательное пространство задается уравнениями:
  \begin{equation}
    \left\{\begin{array}{l}
      x^{k+1} = x_0^{k + 1} + \frac{\delta f'}{\delta x'}(x_0)(x'-x_0') + \ldots + \frac{\delta f^1}{\delta x^k}(x_0)(x^k - x_0^k) \\
      \vdots                                                                                                                       \\
      x^n = x_0^n + \frac{\delta f^{n-k}}{\delta x^1}(x_0)(x' - x_0') + \ldots + \frac{\delta f^{n-k}}{\delta x^k}(x_0)(x^k - x_0^k)
    \end{array}\right.
  \end{equation}

  Пусть
  \begin{equation*}
    \eta = \left(\begin{array}{c}
      \eta'      \\
      \vdots     \\
      \eta^k     \\
      \eta^{k+1} \\
      \vdots     \\
      \eta^n
    \end{array}\right) = \left(\begin{array}{c}
      x^1 - x_0^1         \\
      \vdots              \\
      x^k - x_0^k         \\
      x^{k+1} - x^{k+1}_0 \\
      \vdots              \\
      x^n - x_0^n
    \end{array}\right)
  \end{equation*}

  Тогда система $(1.4)$ примет вид
  \begin{equation}
    \left\{\begin{array}{l}
      \eta^{k+1} = \frac{\delta f'}{\delta x'}(x_0) \cdot \eta' + \ldots + \frac{\delta f'}{\delta x^k}(x_0)\cdot \eta^k \\
      \vdots                                                                                                             \\
      \eta^{n} = \frac{\delta f^{n-k}}{\delta x'}(x_0) \cdot \eta' + \ldots + \frac{\delta f^{n-k}}{\delta x^k}(x_0)\cdot \eta^k
    \end{array}\right.
  \end{equation}

  Таким образом, если вектор $\xi \in TS_{x_0}$, то он полностью определяется своими первыми $k$ координатами, а остальные можно волучить с помощью системы $(1.5)$.

  Построим кривую в $\mathbb{R}^n$, то есть зададим ее уравнением $x = x(t)$:
  \begin{equation}
    l: \ \left\{\begin{array}{l}
      x' = x_0' + \xi't                                     \\
      \vdots                                                \\
      x^k = x_0^k + \xi^k t                                 \\
      x^{k+1} = f^1(x_0' + \xi' t, \ldots, x_0^k + \xi^k t) \\
      \vdots                                                \\
      x^{n} = f^{n-k}(x_0' + \xi' t, \ldots, x_0^k + \xi^k t)
    \end{array}\right., \ v = f(u)
  \end{equation}

  Пусть точка $x_0$ соответствует параметру $t = 0$
  \begin{equation*}
    x(0) = \left\{\begin{array}{l}
      x' = x_0'                        \\
      \vdots                           \\
      x^k = x_0^k                      \\
      x^{k+1} = f^1(x_0',\ldots,x_0^k) \\
      \vdots                           \\
      x^n = f^{n-k}(x_0', \ldots, x_0^k)
    \end{array}\right.,
  \end{equation*}
  то есть кривая проходит через точку $x_0$.

  Далее, функция $f$ удовлетворяет условию $v = f(u) \iff F(u,v) = 0$. Тогда $F(u,f(u)) = 0\implies l$, заданная система $(1.6), \ l \subset S$.

  Таким образом построили гладкий путь, лежащий на поверхности $S$, проходящий через точку $x_0 \in S$, вектор $x'(t_0)$ -- его касательный вектор $\in TS_{x_0}$
\end{proof}

\section{Условный экстремум функции многих переменных}

\textbf{Задача.} Дана функция $u = f(x^1,\ldots,x^n)$ и дана поверхность, заданная уравнениями:
\begin{equation}
  \left\{\begin{array}{l}
    F^1(x^1,\ldots,x^n) = 0 \\
    \vdots                  \\
    F^k(x^1,\ldots,x^n) = 0
  \end{array}\right.
\end{equation}

Нужно найти точку $x_0 = (x_0^1,\ldots,x_0^n)$, в которой
\begin{equation*}
  f(x_0^1,\ldots,x_0^n) = \underset{(\min)}{\max}f(x^1,\ldots,x^n),
\end{equation*}
где $\max \ (\min)$ берется по всем точкам $(x_0^1,\ldots,x_0^n)$, удовлетворяющих уравнениям $(1.7)$.

\begin{center}
  {\large Геометрическая формулировка.}
\end{center}

\textbf{Задача.} Пусть система $(1.7)$ задает в пространстве $\mathbb{R}^n \ m$-мерную поверхность $S$. Найти точку $x_0 \in S:$
\begin{equation*}
  \exists \ U_x(x_0) = U(x_0)\cap S: \quad \forall x \in U_s(x_0)
\end{equation*}
\begin{center}
  $\underset{x_0 - \max}{f(x) \leqslant f(x_0)}$ (или $\underset{x_0 - \min}{f(x) \geqslant f(x_0)}$)
\end{center}

\begin{definition}[линия уровня ($c$-уровень)]
  Пусть $f:D\rightarrow\mathbb{R}, \ D \subset \mathbb{R}^n$ -- область. \textbf{Линией уровня ($c$-уровнем)} функции $f$ называется множество
  \begin{equation*}
    N_c = \{x\in D \ \big| \ f(x) = c\}
  \end{equation*}
\end{definition}

\begin{theorem}[необходимое условие условного локального экстремума]
  Пусть система уровнений
  \begin{equation}
    \left\{\begin{array}{l}
      F^1(x^1,\ldots,x^n) = 0 \\
      \vdots                  \\
      F^{n-k}(x^1,\ldots,x^n) = 0
    \end{array}\right.
  \end{equation}
  задает $(n-k)$-мерную гладкую поверхность $S$ в $D \subset \mathbb{R}^n, \ D$ -- область. Функция $f:D\rightarrow\mathbb{R}$ -- гладкая. Если $x_0 \in S$ является точкой условного локального экстремума для функции $f$, то существует такой набор чисел $\lambda_1,\lambda_2,\ldots,\lambda_{n-k} \in \mathbb{R}:$
  \begin{equation*}
    grad f(x_0) = \sum_{i = 1}^{k}\lambda_i \cdot grad F^i(x_0)
  \end{equation*}
\end{theorem}

\begin{proof}[Доказательство теоремы]
  \begin{lemma}
    Если $x_0$ -- точка условного локального экстремума для функции $f$ и $x_0$ не является критической для функции $f$ (то есть $df(x_0)\ne0$), то касательное пространство $TS_{x_0}\subset TN_{x_0}$, где
    \begin{equation*}
      N_{x_0} = \{x \in D \ \big| \ f(x) = f(x_0)\}
    \end{equation*} -- поверхность уровня, проходящая через $x_0$.
  \end{lemma}

  \begin{proof}[Доказательство леммы]
    Пусть $\xi \in TS_{x_0}$. Пусть $x = x(t)$ -- гладкая кривая на $S: \ x(0) = x_0, \ x'(0) = \xi$.

    Так как точка $x_0$ -- условный экстремум, для функции $f$, то точка $t = 0$ есть локальный экстремум для функции $f(x(t)) \underset{th. \ Fermat's}{\implies}$
    \begin{equation}
      [f(x(t))]_t'(0) = 0 \iff
      f_x'(x_0) \cdot x_t'(0) = 0
    \end{equation}

    Касательное пространство к $N_{x_0}$ в точке $x_0$ имеет уравнение:
    \begin{equation}
      f_x'(x_0)\cdot \xi = 0
    \end{equation}

    Заметим, что $(1.9)$ и $(1.10)$ -- одно и то же уравнение, то есть
    \begin{equation*}
      x_t'(0) = \xi \implies x_t'(0) \in TN_{x_0}
    \end{equation*}
  \end{proof}

  Касательное пространство $TS_{x_0}$ задается уравнениями
  \begin{equation}
    \left\{\begin{array}{l}
      \frac{\delta F^1}{\delta x^1}(x_0)\cdot \xi^1 + \ldots + \frac{\delta F^1}{\delta x^n}(x_0) \cdot \xi^n = 0 \\
      \vdots                                                                                                      \\
      \frac{\delta F^{n-k}}{\delta x^1}(x_0)\cdot \xi^1 + \ldots + \frac{\delta F^{n-k}}{\delta x^n}(x_0) \cdot \xi^n = 0
    \end{array}\right.,
  \end{equation}
  но $\forall i = \overline{1,n-k}:$
  \begin{equation*}
    \bigg\{\frac{\delta F^i}{\delta x^1}\cdot (x_0);\ldots;\frac{\delta F^i}{\delta x^n}\bigg\} = grad F^i(x_0)
  \end{equation*}

  Перепишем $(1.11)$ в виде:
  \begin{equation}
    \left\{\begin{array}{l}
      (grad F^1(x_0),\xi) = 0 \\
      \vdots                  \\
      (grad F^{n-k}(x_0,),\xi) = 0
    \end{array}\right.
  \end{equation}

  Касательное пространство $TN_{x_0}$ к $N_{x_0} = \{x \in D \ \big| \ f(x) = f(x_0)\}$ задается уравнением: $f'(x_0)\cdot\xi = 0$. Заметим, что $f'(x_0) = grad f(x_0) = \\ \big\{\frac{\delta f(x_0)}{\delta x^1},\ldots,\frac{\delta f(x_0)}{\delta x^n}\big\} \implies f'(x_0)\cdot \xi = 0 \iff$
  \begin{equation}
    \iff (frad f(x_0),\xi) = 0
  \end{equation}

  Таким образом из леммы следует, что $\forall \xi$ удовлетворяет системе уравнений $(1.12)$, так же удовлетворяет уравнению $(1.13)$, то есть из того, что $\forall i \in \overline{1,n-k} \ \xi \perp grad F^i(x_0) \implies \xi \perp frad f(x_0) \implies \exists \lambda_1,\ldots,\lambda_{n-k} \in \mathbb{R}:$
  \begin{equation*}
    frad f(x_0) = \sum_{i = 1}^{n-k} \lambda_i\cdot grad F^i(x_0)
  \end{equation*}
\end{proof}

\section*{Метод Лагранжа}

Пусть требуется найти условный экстремум функции $f:D\rightarrow\mathbb{R}, \ D$ -- область в $\mathbb{R}^n$, на поверхности $S$, заданной системой уравнений
\begin{equation*}
  \left\{\begin{array}{l}
    F^1(x^1,\ldots,x^n) = 0 \\
    \vdots                  \\
    F^k(x^1,\ldots,x^n) = 0
  \end{array}\right.
\end{equation*}

Составим функцию Лагранжа:
\begin{multline*}
  L(x,\lambda) = L(x^1,\ldots,x^n,\lambda^1,\ldots,\lambda^k) = \\
  = f(x^1,\ldots,x^n) + \sum_{i=1}^{k}\lambda^i\cdots F^i(x^1,\ldots,x^n),
\end{multline*}

$\lambda = (\lambda^1,\ldots,\lambda^k), \ \lambda^i \in \mathbb{R}$ -- коэффициент, в общем случае пока неизвестен.

Необходимое условие локального экстремума для функции $L:$
\begin{equation}
  \left\{\begin{array}{l}
    \frac{\delta L}{\delta x^1} = \frac{\delta f}{\delta x^1} + \sum_{i = 1}^{k}\lambda^i\cdot\frac{\delta F^i}{\delta x^1} = 0 \\
    \vdots                                                                                                                      \\
    \frac{\delta L}{\delta x^n} = \frac{\delta f}{\delta x^n} + \sum_{i = 1}^{k}\lambda^i\cdot\frac{\delta F^i}{\delta x^n} = 0 \\
    \frac{\delta L}{\delta \lambda^1} = F^1(x^1,\ldots,x^n) = 0                                                                 \\
    \vdots                                                                                                                      \\
    \frac{\delta L}{\delta \lambda^k} = F^k(x^1,\ldots,x^n) = 0
  \end{array}\right.
\end{equation}

\begin{definition}[условный экстремум]
  Пусть $f:D\rightarrow\mathbb{R}, \ D \subset \mathbb{R}^n$ -- область, $S$ -- поверхность в $D$, \textbf{условным экстремумом} функции $f$ называется экстремум функции $f\big|_S$.
\end{definition}

\section*{Достаточное условие условного локального экстремума}

Пусть $f:D\rightarrow\mathbb{R}, \ D \subset \mathbb{R}^n$ -- область, $f \in C^{(2)} \ (D,\mathbb{R}), \ S$ -- $(n-k)$-мерная поверхность в $D$, заданная системой уравнений
\begin{equation*}
  \left\{\begin{array}{l}
    F^1(x^1,\ldots,x^n) = 0 \\
    \vdots                  \\
    F^k(x^1,\ldots,x^n) = 0
  \end{array}\right..
\end{equation*}

Функция Лагранжа
\begin{equation*}
  L(x,\lambda) = f(x^1,\ldots,x^n) + \sum_{i = 0}^{k}\lambda_i \cdot F^i(x^1,\ldots,x^n).
\end{equation*}

Здесь $\lambda_1,\ldots,\lambda_k$ выбираются таким образом, чтобы было выполнено необходимое условие условного экстремума в точке $x_0$.

\begin{theorem}[достаточное условие условного экстремума]
  Если при введенных выше условиях квадратичная форма
  \begin{equation*}
    Q(\xi) = \sum_{i,j=1}^{n}\frac{\delta^2 L}{\delta x^i \delta x^j}(x_0)\cdot\xi^i\cdot\xi^j,
  \end{equation*}
  \begin{equation*}
    (\xi=(\xi^1,\ldots,\xi^n))
  \end{equation*}

  Если:
  \begin{enumerate}
    \item Определена на $TS_{x_0}$
          \begin{enumerate}
            \item Если $Q$ знакоположительная, то точка $x_0$ -- точка условного локального $\min$
            \item Если $Q$ знакоотрицательная, то точка $x_0$ -- точка условного локального $\max$
          \end{enumerate}
    \item Если $Q$ может принимать значения разных знаков, то в точке $x_0$ условного экстремума не наблюдается
  \end{enumerate}
\end{theorem}

\begin{proof}
  Заметим, что $f\big|_S$ и $L\big|_S$ совпадают. В самом деле, если $x\in S$, то
  \begin{equation*}
    L(x,\lambda) = f(x) + \frac{1}{2}\cdot\sum_{i=1}^{k}\lambda_i\cdot F^i(x) = f(x).
  \end{equation*}

  Поэтому покажем, что условие знакопостоянства $Q$ является достаточным для экстремума функции $L\big|_s$.

  Имеем, что
  \begin{equation*}
    \left\{\begin{array}{l}
      \frac{\delta L}{\delta x^1}(x_0) = 0 \\
      \vdots                               \\
      \frac{\delta L}{\delta x^n}(x_0) = 0
    \end{array}\right.
  \end{equation*}

  По формуле Тейлора:
  \begin{equation}
    L\big|_S(x) - L(x_0) = \sum_{i,j=1}^{n}\frac{\delta^2 L(x_0)}{\delta x^i \delta x^j}(x^i - x_0^i)\cdot (x^j - x_0^j) + o(||x-x_0||^2)
  \end{equation}

  Так как $S$ -- $m$-мерная ($m=n-k$) поверхность, то существует гладкое отображение $x(t):\mathbb{R}^m\rightarrow\mathbb{R}^n: \ x = x(t) \subset S \ \forall t \in \mathbb{R}^m, \ x(0) = x_0$. Отображение $x(t)$ биективно отображает $\mathbb{R^m}$ на $U_S(x_0) = U(x_0)\cap S$.

  Если $x\in S$, то условие дифференцируемости $x(t)$:
  \begin{equation*}
    x-x_0 = x(t) - x(0) = x'(0)\cdot t + o(||t||)
  \end{equation*}
  \begin{center}
    или
  \end{center}
  \begin{equation*}
    \left\{\begin{array}{l}
      x^1 - x^1_0 = \frac{\delta x^1}{\delta t^1}(0)\cdot t^1 + \ldots + \frac{\delta x^1}{\delta t^m}(0)\cdot t^m + d(||t||) \\
      \vdots                                                                                                                  \\
      x^n - x^n_0 = \frac{\delta x^n}{\delta t^1}(0)\cdot t^1 + \ldots + \frac{\delta x^n}{\delta t^m}(0)\cdot t^m + d(||t||) \\
    \end{array}\right.
  \end{equation*}
  \begin{center}
    или кратко
  \end{center}
  \begin{equation}
    \left\{\begin{array}{l}
      x^1 - x^1_0 = \sum_{i=1}^{m}\frac{\delta x^1}{\delta t^i}(0)\cdot t^i + o(||t||) \\
      \vdots                                                                           \\
      x^n - x^n_0 = \sum_{i=1}^{m}\frac{\delta x^n}{\delta t^i}(0)\cdot t^i + o(||t||) \\
    \end{array}\right.
  \end{equation}

  Подставим $(1.16)$ в $(1.15)$:
  \begin{multline*}
    L\big|_S(x) - L(x_0) = \frac{1}{2}\cdot\sum_{i,j=1}^{n}\frac{\delta^2L(x_0)}{\delta x^i \delta x^j} \cdot(\sum_{\alpha=1}^{m}\frac{\delta x^i}{\delta t^\alpha}(0)\cdot t^\alpha + o(||t||))\cdot \\
    \cdot(\sum_{\beta=1}^{m}\frac{\delta x^j}{\delta t^\beta}(0)\cdot t^\beta + o(||t||)) + o(||x-x_0||^2) \overset{(*)}{=} \\
    = \frac{1}{2}\sum_{i,j=1}^{n}\frac{\delta^2 L(x_0)}{\delta x^i \delta x^j}\cdot\big[(\sum_{\alpha=1}^{m}\frac{\delta x^i}{\delta t^\alpha}(0)\cdot t^\alpha)\cdot(\sum_{\beta=1}^{m}\frac{\delta x^i}{\delta t^\beta}(0)\cdot t^\beta) + \\
      + (\sum_{\alpha=1}^{m}\frac{\delta x^i}{\delta t^\alpha}(0)t^\alpha)\cdot o(||t||) + (\sum_{\beta=1}^{m}\frac{\delta x^i}{\delta t^\beta}(0)\cdot t^\beta)\cdot \\
      \cdot o(||t||) + o(||t||)\big] + o(||x-x_0||^2) = \frac{1}{2}\sum_{i,j=1}^{n}\frac{\delta^2 L(x_0)}{\delta x^i \delta x^j} \cdot \\
    \cdot \sum_{\alpha,\beta = 1}^{m}\frac{\delta x^i}{\delta t^\alpha} \cdot \frac{x^i}{\delta t^\beta} \cdot t^\alpha \cdot t^\beta + o(||t||^2) = \frac{||t||^2}{2} \cdot \sum_{i,j=1}^{n}\frac{\delta^2 L(x_0)}{\delta x^i\delta x^j} \cdot \\
    \cdot \sum_{\alpha,\beta=1}^{m}\frac{\delta x^i}{\delta t^\alpha}\cdot \frac{\delta x^j}{\delta t^\beta} \cdot \frac{t^\alpha}{||t||} \cdot \frac{t^\beta}{||t||} + o(||t||^2) = \frac{||t||^2}{2}Q(\xi) + o(||t||^2).
  \end{multline*}

  Таким образом получаем, что
  \begin{equation*}
    L\big|_S(x) - L(x_0) = \frac{||t||^2}{2} \cdot Q(\xi) + o(||t||^2), \ \xi \in TS_{x_0}.
  \end{equation*}

  Тогда, если $Q> 0$, то
  \begin{equation*}
    L\big|_S(x) - L(x_0)> 0 \implies x_0 \ \min \ L\big|_S(x) \implies x_0 \ \min \ f\big|_S.
  \end{equation*}

  Если $Q < 0$, то
  \begin{equation*}
    L\big|_S(x) - L(x_0) < 0 \implies x_0 \ \max \ L\big|_S(x) \implies x_0 \ \max \ f\big|_S \ (\forall x \in U_S(x_0))
  \end{equation*}

  Если $Q$ -- знакопеременна, то $x(t)$ не для всех $x \in U_S(x_0)$ разность $L\big|_S(x) - L(x_0)$ имеет постоянный знак $\implies$ в этом случае в точке $x_0$ нет экстремума.

  Докажем $(*)$, то есть покажем, что
  \begin{equation*}
    o(||t||) \cdot \sum_{\alpha = 1}^{m} \frac{\delta x^i}{\delta t^\alpha}\cdot t^\alpha = o(||t||^2)
  \end{equation*}
  \begin{center}
    и
  \end{center}
  \begin{equation*}
    o(||x-x_0||^2) = o(||t||^2), \ x \in S.
  \end{equation*}

  В самом деле,
  \begin{equation*}
    \bigg| \sum_{\alpha=1}^{m}\frac{\delta x^i}{\delta t^\alpha}(0)\cdot t^\alpha\bigg| \leqslant \sum_{\alpha=1}^{m}\bigg|\frac{\delta x^i}{\delta t^\alpha}(0)\bigg| \cdot \big|t^\alpha \big| \leqslant ||t|| \cdot \sum_{\alpha = 1}^{m} \bigg|\frac{\delta x^i}{\delta t^\alpha}(0)\bigg| = A\cdot ||t||
  \end{equation*}

  Таким образом,
  \begin{equation*}
    o(||t||) \cdot \bigg|\sum_{\alpha = 1}^{m}\frac{\delta x^i(0)}{\delta t^\alpha}\cdot t^\alpha \bigg| \leqslant o(||t||)\cdot O(||t||) = \omega(t)\cdot ||t|| \cdot \gamma(t) \cdot ||t|| =
  \end{equation*}
  \begin{center}
    $= \bigg|$ где $\omega(t)\rightarrow0$ при $t\rightarrow 0$, $\gamma(t)$ -- ограниченная функция $\bigg| =$
  \end{center}
  \begin{equation*}
    = \alpha(t) \cdot ||t||^2 = o(||t||^2), \ \alpha(t)\rightarrow 0, \ t\rightarrow 0
  \end{equation*}

  Далее, если $x\in S$, то
  \begin{multline*}
    ||x-x_0||^2 = \left|\left|\left(\begin{array}{c}
        x^1 - x_0^1 \\
        \vdots      \\
        x^n - x^n_0
      \end{array}\right)\right|\right|^2 \overset{(1.16)}{=} \left|\left|\left(\begin{array}{c}
        \sum_{\alpha=1}^{m}\frac{\delta x^1}{\delta t^\alpha}\cdot t^\alpha + \ldots \\
        \vdots                                                                       \\
        \sum_{\alpha=1}^{m}\frac{\delta x^n}{\delta t^\alpha}\cdot t^\alpha + \ldots
      \end{array}\right)\right|\right|^2 = \\
    = \bigg(\sum_{\alpha = 1}^{m}\frac{\delta x^1}{\delta t^\alpha}\cdot t^\alpha + o(||t||)\bigg)^2 + \ldots + \bigg(\sum_{\alpha=1}^{m}\frac{\delta x^n}{\delta t^\alpha} + o(||t||)\bigg)^2 =\\
    = \bigg(\sum_{\alpha = 1}^{m}\frac{\delta x^1}{\delta t^\alpha}\cdot t^\alpha \bigg)^2 + \ldots + \bigg(\sum_{\alpha=1}^{m}\frac{\delta x^n}{\delta t^\alpha} \bigg)^2 + o(||t||^2) \leqslant \\
    \leqslant \bigg( \underset{\alpha}{\max}\frac{\delta x^1}{\delta t^\alpha}\bigg)^2 \cdot \bigg( \frac{\alpha=1}{m}t^\alpha\bigg)^2 + \ldots + \bigg(\underset{\alpha}{\max}(\frac{\delta x^n}{\delta t^\alpha}) \bigg)^2\cdot(\sum_{\alpha=1}^{n}t^\alpha)^2 \leqslant \\
    \leqslant ||t||^2 \cdot
  \end{multline*}

  ДОПИСАТЬ
\end{proof}

\chapter{Теория рядов}

\section{Числовые ряды}

\begin{definition}[ряд, член ряда, $n$-мерный член ряда, частичная сумма]
  \textbf{Рядом} называется выражение
  \begin{equation*}
    a_1 + a_2 + \ldots + a_n + \ldots,
  \end{equation*}
  где $a_i \in \mathbb{R}$.

  Числа $a_i$ называются \textbf{членами ряда}, $a_n$ -- \textbf{$n$-мерным членом ряда}.
  \begin{equation}
    \sum_{n=1}^{\infty}a_n
  \end{equation}

  Рассмотрим числа: $\quad \begin{array}{l}
      A_1 = a_1       \\
      A_2 = a_1 + a_2 \\
      \vdots          \\
      A_n = a_1 + a_2 + \ldots + a_n
    \end{array}$

  Числа $A_1,A_2,\ldots,A_n$ называются \textbf{частичными суммами} ряда $(2.1)$
\end{definition}

\begin{definition}
  Говорят, что ряд $(2.1)$ сходится, если существует конечный предел частичных сумм, то есть
  \begin{equation*}
    \exists \underset{n\rightarrow\infty}{\lim}A_n = A.
  \end{equation*}

  Тогда сумма бесконечного ряда $(2.1)$ полагается равной
  \begin{equation*}
    A = \sum_{n=1}^{\infty}a_n.
  \end{equation*}
\end{definition}

\begin{example}
  \item \begin{equation*}
    10 + 1 + \frac{1}{10} + \frac{1}{10^2} + \ldots + \frac{1}{10^n} + \ldots = 10 + \sum_{k=0}^{\infty}\frac{1}{10^k}
  \end{equation*}
  \begin{multline*}
    A_n = \frac{1}{10^0} + \frac{1}{10^1} + \ldots + \frac{1}{10^n} = \frac{1\cdot(q^k - 1)}{q - 1} = \\
    = \frac{\frac{1}{10^n} - 1}{\frac{1}{10} - 1} = \frac{1 - \frac{1}{10^n}}{\frac{9}{10}} = \frac{10}{9} \cdot (1 - \frac{1}{10^n})
  \end{multline*}
  \begin{equation*}
    \underset{n\rightarrow\infty}{\lim}A_n = \underset{n\rightarrow\infty}{\lim}\frac{10}{9}(1-\frac{1}{10^n}) = \frac{10}{9}
  \end{equation*}
\end{example}

\subsection{Гармонический ряд}

\begin{definition}[среднее гармоническое]
  Число $C$ называется \textbf{средним гармоническим} чисел $a$ и $b$ ($a,b \ne 0$), если
  \begin{equation*}
    \frac{1}{c} = \frac{1}{2}\cdot (\frac{1}{a} + \frac{1}{b})
  \end{equation*}
\end{definition}

\begin{definition}
  Ряд вида
  \begin{equation}
    \sum_{n=1}^{\infty}\frac{1}{n}
  \end{equation}
  называется \textbf{гармоническим}.

  Покажем, что $(2.2)$ расходится.

  В самом деле,
  \begin{equation*}
    \underbrace{1}_{>\frac{1}{2}} + \underbrace{\frac{1}{2}}_{\geqslant \frac{1}{2}} + \underbrace{\frac{1}{3} + \frac{1}{4}}_{> \frac{1}{2}} + \underbrace{\frac{1}{5} + \frac{1}{6} + \frac{1}{7}}_{> \frac{1}{2}} + \underbrace{\frac{1}{8} + \frac{1}{9} + \frac{1}{10} + \frac{1}{11} + \frac{1}{12}}_{> \frac{1}{2}} + \ldots
  \end{equation*}
  \begin{eqnarray*}
    A_n = 1 + \frac{1}{2} + \frac{1}{3} + &\ldots& + \frac{1}{n} \\
    \forall E &>& 0 \ \exists N : \ \forall n > N \quad |A_n| > E
  \end{eqnarray*}
  \begin{equation*}
    \underset{n\rightarrow\infty}{\lim}A_n = \infty
  \end{equation*}
\end{definition}

\subsection{Основные свойства сходящихся рядов}

\begin{theorem}[критерий Коши]
  Ряд $(2.1)$
  \begin{equation*}
    \sum_{n=1}^{\infty}a_n
  \end{equation*}
  сходится тогда и только тогда, когда $\forall \epsilon > 0 \ \exists N \in \mathbb{N}: \ \forall n > N, \ \forall p > 0$
  \begin{equation*}
    |a_{n+1} + \ldots + a_{n+p}| < \epsilon
  \end{equation*}
\end{theorem}

\begin{proof}
  Ряд $\sum_{n=1}^{\infty}$ сходится $\underset{by \ def.}{\iff} \exists \underset{n\rightarrow}{\lim}A_n \iff A_n$ -- фундаментальная последовательность: $\forall \epsilon > 0 \ \exists N \in \mathbb{N}: \ \forall n > N$ и $\forall p > 0$
  \begin{equation*}
    |A_n - A_{n+p}| < \epsilon
  \end{equation*}
  \begin{center}
    (критерий Коши сходимости последовательности)
  \end{center}
  Имеем
  \begin{multline*}
    |A_n - A_{n+p}| = \\
    =|a_1 + a_2 + \ldots + a_n - (a_1 + a_2 + \ldots + a_n + \ldots + a_{n+p})| = \\
    = |a_{n+1} + \ldots + a_{n+p}| < \epsilon.
  \end{multline*}
\end{proof}

\begin{example}
  Докажем, что ряд $(2.2)$ расходится.

  Если $\exists\epsilon > 0 \ \forall N \ \exists n > N \ \exists p > 0$
  \begin{equation*}
    |a_{n+1} + \ldots + a_{n+p}| \geqslant \epsilon
  \end{equation*}
  \begin{multline*}
    |\frac{1}{n+1} + \frac{1}{n+2} + \ldots + \frac{1}{n+p}| \geqslant |\frac{1}{n+p} + \frac{1}{n+p} + \ldots + \frac{1}{n+p}| = \\
    = \frac{p}{n+p} = \left| n = p \right| = \frac{1}{2},
  \end{multline*}
  то есть для $\forall N: \ \epsilon = \frac{1}{2} \quad p = n, \ n = N + 1 \implies$ по критерию Коши, гармонический ряд $(2.2)$ расходится.
\end{example}

\begin{remark}
  Со всякой последовательностью $x_n$ можно связать ряд, частичными суммами которого являются члены этой последовательности. Пусть:
  \begin{equation*}
    x_1,x_2,\ldots,x_n,\ldots
  \end{equation*}

  Тогда ряд
  \begin{equation*}
    \underbrace{x_1}_{a_1} + \underbrace{(x_2 - x_1)}_{a_2} + \underbrace{(x_3 - x_2)}_{a_3} + \ldots + \underbrace{(x_n - x_{n-1})}_{a_n} + \ldots
  \end{equation*}
  \begin{equation*}
    A_n = a_1 + \ldots + a_n = x_1 + (x_2 - x_1) + \ldots + (x_n - x_{n-1}) = x_n
  \end{equation*}
\end{remark}

\begin{theorem}[необходимое условие сходимости ряда]
  Если ряд $(2.1)$
  \begin{equation*}
    \sum_{n=1}^{\infty}a_n
  \end{equation*}
  сходится, тогда
  \begin{equation*}
    \underset{n\rightarrow\infty}{\lim}a_n = 0
  \end{equation*}
\end{theorem}

\begin{proof}
  Пусть ряд $(2.1)$
  \begin{equation*}
    \sum_{n=1}^{\infty}a_n
  \end{equation*}
  сходится, тогда $\exists \underset{n\rightarrow\infty}{\lim}A_n$,
  \begin{eqnarray*}
    \underset{n\rightarrow\infty}{\lim}a_n &=& \underset{n\rightarrow\infty}{\lim}(A_n - A_{n-1}) = \\
    = \underset{n\rightarrow\infty}{\lim}A_n - \underset{n\rightarrow\infty}{\lim}A_{n-1} &=& 0
  \end{eqnarray*}
\end{proof}

\begin{definition}[$m$-ный остаток]
  Пусть дан ряд $(2.1)$
  \begin{equation*}
    \sum_{n=1}^{\infty}a_n.
  \end{equation*}

  Ряд вида
  \begin{equation}
    \sum_{n=m+1}^{\infty}a_n
  \end{equation}
  называется \textbf{$m$-ным остатком} ряда $(2.1)$
\end{definition}

\begin{theorem}[об остатке ряда]
  Следующие условия эквивалентны:
  \begin{enumerate}
    \item Ряд $(2.1)$ сходится
    \item $\forall$ его состаток сходится
    \item Некоторый его остаток $(2.2)$ сходится
  \end{enumerate}
\end{theorem}

\begin{proof}
  \begin{itemize}
    \item $1.\implies 2.$

          Пусть ряд $(2.1)$ сходится и его сумма равна $A$.

          Пусть
          \begin{equation*}
            A_k^* = \sum_{n=m+1}^{m+k}a_n
          \end{equation*}
          -- $k$-тая частичная сумма ряда $(2.2)$.

          Ряд $(2.2)$ сходится, если $\exists \underset{k\rightarrow\infty}{A_k^*}$:
          \begin{equation*}
            A_k^* = A_{m+k} - A_m
          \end{equation*}
          \begin{eqnarray*}
            \underset{k\rightarrow\infty}{\lim}A_k^* &=& \underset{k\rightarrow\infty}{\lim}(A_{m+k} - A_n) = \\
            = \underset{k\rightarrow\infty}{\lim}A_{m+k} - \underset{k\rightarrow\infty}{\lim}A_{m} &=& A - A_m
          \end{eqnarray*}

    \item $2. \implies 3.$ -- очевидно
    \item $3. \implies 1.$

          Пусть ряд $(2.1)$
          \begin{equation*}
            \sum_{n=m+1}^{\infty}A_n
          \end{equation*}
          -- сходится.

          Тогда при $n > m$
          \begin{equation*}
            A_n = A_m + A_{n-m}^* = \sum_{k=m+1}^{m + (n-m)}a_k
          \end{equation*}

          Ряд $(2.1)$
          \begin{equation*}
            \sum_{n=1}^{\infty}a_n
          \end{equation*}
          сходится тогда и только тогда (по определению), когда
          \begin{equation*}
            \exists \underset{n\rightarrow\infty}{\lim}A_n.
          \end{equation*}

          Рассмотрим
          \begin{eqnarray*}
            \underset{n\rightarrow\infty}{\lim}A_n &=& \underset{n\rightarrow\infty}{\lim}(A_m + A_{n-m}^*) = \\
            &=& \underset{n\rightarrow\infty}{\lim}A_m + \underset{n\rightarrow\infty}{\lim}A_{n-m}
          \end{eqnarray*}
          $\implies \exists \underset{n\rightarrow\infty}{\lim}A_n \implies (2.1)$ -- сходится.
  \end{itemize}
\end{proof}

Обозначим $\alpha_m$ -- сумма $m$-того остатка ряда $(2.1) \ =$ сумме ряда $(2.2)$
\begin{equation*}
  \alpha_m = \sum_{n=m+1}^{\infty}a_n
\end{equation*}
\begin{center}
  ($(2.1)$ сходится в этом случае)
\end{center}

\begin{effect}
  Ряд $(2.1)$ сходится $\iff \underset{m\rightarrow\infty}{\lim} \alpha_m = 0$
\end{effect}

\begin{proof}
  Самостоятельно $\overset{\cdot \lor \cdot}{\smile}$
\end{proof}

\begin{definition}
  Пусть даны ряды
  \begin{equation*}
    (A) \ \sum_{n=1}^{\infty}a_n; \quad (B) \ \sum_{n=1}^{\infty}b_n
  \end{equation*}

  ДОПИСАТЬ
\end{definition}

\begin{theorem}[1-ый признак сравнения]
  Пусть даны ряды
  \begin{equation*}
    (A) \ \sum_{n=1}^{\infty}a_n; \quad (B) \ \sum_{n=1}^{\infty}b_n
  \end{equation*}

  Причем $a_n > 0, \ b_n > 0 \ \forall n$.

  Если $\exists N \in \mathbb{N}: \ \forall n > N \quad a_n \leqslant b_n$, то
  \begin{enumerate}
    \item Из сходимости ряда $(B) \implies$ сходимость ряда $(A)$
    \item Из сходимости ряда $(A) \implies$ сходимость ряда $(B)$
  \end{enumerate}
\end{theorem}

\begin{proof}
  \begin{enumerate}
    \item Пусть ряд $(B)$ -- сходится $\implies$ по теореме $2.2.1$ его частичные суммы ограничены $\implies$ по неравенству $a_n\leqslant b_n$ частичные суммы ряда $(A)$ также ограничены $\implies$ по $2.2.1$ ряд $(A)$ сходится.
    \item Аналогично
  \end{enumerate}
\end{proof}

\begin{theorem}[2-ой признак сравнения]
  Пусть даны ряды
  \begin{equation*}
    (A) \ \sum_{n=1}^{\infty}a_n; \quad (B) \ \sum_{n=1}^{\infty}b_n
  \end{equation*}

  Причем $a_n > 0, \ b_n > 0 \ \forall n$.

  Если $\underset{n\rightarrow\infty}{\lim}\frac{a_n}{b_n}=k, \ k\in [0;\infty]$, то
  \begin{enumerate}
    \item При $k=\infty$ из сходимости $(A) \implies$ сходимость ряда $(B)$
    \item При $k=0$ из сходимости ряда $(B) \implies$ сходимость ряда $(A)$
    \item При $0<k<\infty \ (k = const \ne 0)$ ряды $(A)$ и $(B)$ ведут себя одинаково
  \end{enumerate}
\end{theorem}

\begin{proof}
  Переписать доказательство для несобственных интегралов, заменив слово "интеграл" на слово "ряд".
\end{proof}

\begin{theorem}[3-й признак сравнения]
  Пусть даны ряды
  \begin{equation*}
    (A) \ \sum_{n=1}^{\infty}a_n; \quad (B) \ \sum_{n=1}^{\infty}b_n
  \end{equation*}

  Причем $a_n > 0, \ b_n > 0 \ \forall n$.

  Если $\exists N \in \mathbb{N}: \ \forall n > N$
  \begin{equation*}
    \frac{a_{n+1}}{a_n}\leqslant\frac{b_{n+1}}{b_n},
  \end{equation*}
  то
  \begin{enumerate}
    \item Из сходимости ряда $(B) \implies$ сходимость ряда $(A)$
    \item Из расходимости ряда $(A) \implies$ расходимость ряда $(B)$
  \end{enumerate}
\end{theorem}

\begin{proof}
  Можно считать, что $N = 0$.

  Тогда $\forall n > N$ имеем:
  \begin{equation*}
    \frac{a_2}{a_1}\leqslant\frac{b_2}{b_1}; \quad \frac{a_3}{a_2}\leqslant\frac{b_3}{b_2}; \quad \frac{a_4}{a_3}\leqslant\frac{b_4}{b_3}; \quad \ldots; \quad \frac{a_{n+1}}{a_n} \leqslant \frac{b_{n+1}}{b_n}.
  \end{equation*}

  Перемножим левые и правые части:
  \begin{equation*}
    \frac{a_2 \cdot a_3 \cdot a_4 \cdot \ldots \cdot a_{n+1}}{a_1 \cdot a_2 \cdot a_3 \cdot \ldots \cdot a_n} \leqslant \frac{b_2 \cdot b_3 \cdot b_4 \cdot \ldots \cdot b_{n+1}}{b_1 \cdot b_2 \cdot b_3 \cdot \ldots \cdot b_n},
  \end{equation*}
  \begin{equation*}
    \frac{a_{n+1}}{a_1} \leqslant \frac{b_{n+1}}{b_n} \implies a_{n+1} \leqslant \frac{a_1}{b_1}\cdot b_{n+1}.
  \end{equation*}
  \begin{enumerate}
    \item Если ряд $(B) \ \sum_{n=1}^{\infty}b_n$ сходится $\implies$ сходится ряд $\sum_{n=1}^{\infty}\frac{a_1}{b_1}\cdot b_{n+1} \implies$ сходится ряд $\sum_{n=1}^{\infty}a_{n+1} \implies \sum_{n=1}^{\infty}a_n$
    \item Аналогично
  \end{enumerate}
\end{proof}

\begin{theorem}[интегральный признак Коши-Маклорена]
  Пусть дан положительный ряд
  \begin{equation*}
    (A) \ \sum_{n=1}^{\infty}a_n
  \end{equation*}

  Если функция $f(x)$ удовлетворяет следующим условиям:
  \begin{enumerate}
    \item $f(x): \ [1;+\infty] \rightarrow\mathbb{R}$
    \item $f(x)$ -- монотонна
    \item $f(x) = a_n, \ \forall n \in \mathbb{N}$
  \end{enumerate}
  тогда ряд $(A)$ и интеграл $\int_{1}^{\infty}f(x)dx$ ведут себя одинаково
\end{theorem}

\begin{proof}
  Ограничимся случаем, когда $f(x)$ монотонно убывает.

  Рассмотрим функцию $\phi(x) = a_n$ при $n \leqslant x < n+1$ и $\psi(x) = a_{n+1}$ при $n\leqslant x <n + 1$. Тогда $\forall x \in [1;+\infty)$
  \begin{equation*}
    \psi(x) \leqslant f(x) \leqslant\phi(x).
  \end{equation*}

  Отсюда
  \begin{multline*}
    \int_{1}^{N}\psi(x)dx \leqslant \int_{1}^{N}f(x)dx \leqslant\int_{1}^{N}\phi(x)dx \implies \\
    \implies \underbrace{\sum_{n=1}^{N}a_{n+1}}_{partial \ series \ sum \ (A)}^{(1)} \leqslant \int_{1}^{N}f(x)dx \leqslant \underbrace{\sum_{n=1}^{N}a_n}_{partial \ series \ sum \ (A)}^{(2)}
  \end{multline*}

  Если интеграл сходится, то частичная сумма $(1)$ ограничена $\implies$ ряд $(A)$ сходится. Если интеграл расзодится, то частичная сумма $(2)$ непрерывна $\implies$ ряд $(A)$ -- расходится.

  Если ряд $(A)$ сходится, то $(2)$ -- ограничена $\implies$ $\int_{1}^{N}f(x)dx$ -- ограничен $\implies \int_{1}^{\infty}f(x)dx$ -- сходится.

  Если ряд $(A)$ расходится $\implies$ частичная сумма $(1)$ неограничена $\implies$ интеграл расходится.
\end{proof}

\begin{example}
  \begin{enumerate}
    \item $\sum_{n=1}^{\infty}\frac{1}{n^p}$

          Рассмотрим $f(x) = \frac{1}{x^p}$ на $[1;+\infty)$ -- непрерывно монотонно $\downarrow$, $f(n) = \frac{1}{n^p}$.

          $\sum_{n=1}^{\infty}\frac{1}{n^p}$ ведет себя одинаково с интегралом $\int_{1}^{\infty}\frac{dx}{x^p}$ -- сходится при $p>1$ и расходится при $p\leqslant1 \implies$ ряд $\sum_{n=1}^{\infty}\frac{1}{n^p}$ сходится при $p>1$ и расходится при $p\leqslant1$.

    \item $\sum_{n=1}^{\infty}$

          $f(x) = \frac{1}{x\ln x}, \ x \in [e;+\infty), \ \downarrow$, непрерывна.

          \begin{multline*}
            \int_{e}^{\infty}\frac{dx}{x\ln x} = \underset{b\rightarrow\infty}{\lim}\int_{e}^{b}\frac{d(\ln x)}{\ln x} =\\
            = \underset{b\rightarrow\infty}{\lim}(\ln(\ln x))\big|_e^b = \underset{b\rightarrow\infty}{\lim}\ln (\ln b) = \infty \implies
          \end{multline*}
          $\implies$ ряд $\sum_{n=1}^{\infty}\frac{1}{n\ln n}$ расходится (по интегралу Коши-Маклорена)
  \end{enumerate}
\end{example}

\begin{theorem}[радикальный признак Коши]
  Пусть ряд $(A) \ \sum_{n=1}^{\infty}a_n$ положительный и $\underset{n\rightarrow\infty}{\overline{\lim}}\sqrt[n]{a_n} = q$. Тогда:
  \begin{enumerate}
    \item При $q < 1$ ряд $(A)$ сходится
    \item При $q > 1$ ряд $(A)$ расходится
    \item При $q = 1$ -- ?
  \end{enumerate}
\end{theorem}

\begin{proof}
  \begin{enumerate}
    \item Пусть $q < 1$. Возьмем число $r: \ q < r < 1$. Тогда $\exists N: \ \forall n > N$
          \begin{equation*}
            \sqrt[n]{a_n} < r \implies a_n < r^n
          \end{equation*}

          $0 < r < 1 \implies \sum_{n=1}^{\infty}r^n$ -- сходится $\implies$ по 1-му признаку сравнения сходится ряд $\sum_{n=1}^{\infty}a_n$

    \item Пусть $q > 1$, то существует подпоследовательность $\sqrt[n_i]{a_{n_i}} \rightarrow q$ при $i\rightarrow\infty \implies a_{n_i}\rightarrow q^{n_i} > 1 \implies$ ряда $\sum_{n=1}^{\infty}a_n$ расходится

    \item Рассмотрим ряды $\sum_{n=1}^{\infty}\frac{1}{n}$ и $\sum_{n=1}^{\infty}\frac{1}{n^2}$ сходятся.
          \begin{equation*}
            \underset{n\rightarrow\infty}{\lim}\sqrt[n]{\frac{1}{n}} = \underset{n\rightarrow\infty}{\lim}\sqrt[n]{\frac{1}{n^2}} = 1
          \end{equation*}
  \end{enumerate}
\end{proof}

\begin{theorem}[признак Даламбера]
  Пусть ряд $(A) \ \sum_{n=1}^{\infty}a_n$ положительный и
  \begin{equation*}
    \underset{n\rightarrow\infty}{\lim}\frac{a_{n+1}}{a_n} = d.
  \end{equation*}

  Тогда
  \begin{enumerate}
    \item При $d < 1$ ряд $(A)$ сходится
    \item При $d > 1$ ряд $(A)$ расходится
    \item При $d = 1$ -- ?
  \end{enumerate}
\end{theorem}

\begin{proof}
  \begin{enumerate}
    \item Пусть $d < 1$. Возьмем $d < r < 1 \implies \exists N: \ \forall n > N$
          \begin{equation*}
            \frac{a_{n+1}}{a_n}<r
          \end{equation*}
          \begin{equation*}
            b_1 = \frac{a_2}{a_1}; \quad b_2 = \frac{a_3}{a_2}; \quad b_3 = \frac{a_4}{a_3}; \quad \ldots; \quad b_n = \frac{a_{n+1}}{a_n}; \quad \ldots
          \end{equation*}
          Можно считать, что $N=0$, тогда $\forall n > N \begin{array}{l}
              a_2 < r \cdot a_1                 \\
              a_3 < r \cdot a_2 < r^2 \cdot a_1 \\
              a_4 < r \cdot a_3 < r^3 \cdot a_1 \\
              \vdots                            \\
              a_{n+1} < r^n \cdot a_1
            \end{array}$.

          Так как $0 < r < 1$, то $\sum_{n=1}^{\infty} r^n \cdot a_1$ сходится $\implies$ сходится ряд $(A)$ по 1 признаку сравнения.
    \item Самостоятельно.
    \item $\sum_{n=1}^{\infty}\frac{1}{n}, \quad \sum_{n=1}^{\infty}\frac{1}{n^2}$.

          \begin{eqnarray*}
            \underset{n\rightarrow\infty}{\lim}\frac{a_{n+1}}{a_n} = \underset{n\rightarrow\infty}{\lim}\frac{\frac{1}{n+1}}{\frac{1}{n}} & = & \\
            & = & \underset{n\rightarrow\infty}{\lim}\frac{n}{n+1} = 1;
          \end{eqnarray*}

          \begin{equation*}
            \underset{n\rightarrow\infty}{\lim}\frac{\frac{1}{(n+1)^2}}{\frac{1}{n^2}} = \underset{n\rightarrow\infty}{\lim}\frac{n^2}{(n+1)^2} = 1.
          \end{equation*}
  \end{enumerate}
\end{proof}

\begin{theorem}[признак Раббе]
  Пусть $(A) \ \sum_{n=1}^{\infty}a_n$ -- положительная. Если
  \begin{equation*}
    \underset{n\rightarrow\infty}{\lim}n \cdot (\frac{a_n}{a_{n+1}}) = r,
  \end{equation*}
  то
  \begin{enumerate}
    \item При $r>1$ ряд $(A)$ сходится;
    \item При $r<1$ ряд $(A)$ расходится;
    \item При $r=1$ ряд $(A)$ -- ?
  \end{enumerate}
\end{theorem}

\begin{proof}
  \begin{enumerate}
    \item Пусть $r>1$. Возьмем $p$ и $q$:
          \begin{equation*}
            1 < p < q < r.
          \end{equation*}

          Так как
          \begin{equation*}
            \underset{n\rightarrow\infty}{\lim}n\cdot(\frac{a_n}{a_{n+1}} - 1) = r,
          \end{equation*}
          то $\exists N_1: \ \forall n > N_1 \quad n\cdot(\frac{a_n}{a_{n+1}} - 1) > q$, то есть
          \begin{equation}
            \frac{a_n}{a_{n+1}} > 1 + \frac{q}{n}.
          \end{equation}

          Далее, рассмотрим
          \begin{equation*}
            \underset{n\rightarrow\infty}{\lim}\frac{(1 + \frac{1}{n})^p - 1}{\frac{1}{n}} \overset{Taylor's \ f.}{=} \underset{n\rightarrow\infty}{\lim}\frac{1 + \frac{p}{n} + o(\frac{1}{n}) - 1}{\frac{1}{n}} = p < q \implies
          \end{equation*}
          $\implies \exists N_2: \ \forall n > N_2$
          \begin{equation}
            \frac{(1 + \frac{1}{n})^p - 1}{\frac{1}{n}} < q \implies (1 + \frac{1}{n})^p < 1 + \frac{q}{n}.
          \end{equation}

          Сравниваем неравенства $(2.4)$ и $(2.5)$, получим, что при $n > \max(N_1,N_2)$:
          \begin{eqnarray*}
            (1 + \frac{1}{n})^p < 1 + \frac{q}{n} & < & \frac{a_n}{a_{n+1}} \implies \\
            \implies \frac{a_n}{a_{n+1}} & > & (1 + \frac{1}{n}^p) = \frac{(n+1)^p}{n^p} = \frac{\frac{1}{n^p}}{\frac{1}{(n+1)^p}}.
          \end{eqnarray*}

          Ряд $\sum_{n=1}^{\infty}\frac{1}{n^p}$ сходится при $p > 1$:
          \begin{eqnarray*}
            \frac{a_n}{a_{n+1}} > \frac{\frac{1}{n^p}}{\frac{1}{(n+1)^p}} & \implies & a_n \cdot \frac{1}{(n+1)^p} > \\
            > \frac{1}{n^p} \cdot a_{n+1} & \implies & \frac{a_{n+1}}{a_n} < \frac{\frac{1}{(n+1)^p}}{\frac{1}{n^p}}.
          \end{eqnarray*}

          По 3-му признаку сравнения, ряд $(A)$ сходится при $p > 1 \implies$ при $r > 1$.

    \item Пусть $r < 1$. Тогда $\exists N: \ \forall n > N$:
          \begin{multline*}
            n \cdot (\frac{a_n}{a_{n+1}} - 1) < 1 \implies \\
            \implies \frac{a_n}{a_{n+1}} < 1 + \frac{1}{n} = \frac{n+1}{n} = \frac{\frac{1}{n}}{\frac{1}{n+1}} \implies \\
            \implies \frac{a_{n+1}}{a_n} > \frac{\frac{1}{n+1}}{\frac{1}{n}}.
          \end{multline*}

          Ряд $\sum_{n=1}^{\infty}\frac{1}{n}$ -- гармонический, расходящийся $\implies$ по 3-му признаку сравнения ряд $(A)$ расходится.

    \item \underline{Упражнение:} привести 2 примера рядов (сходящийся, расходящийся), но $r=1$ в обоих случаях.
  \end{enumerate}
\end{proof}

\begin{theorem}[признак Кумера]
  Пусть дан ряд $(A) \ \sum_{n=1}^{\infty}a_n$ -- положительный. Пусть числа $c_1,c_2,\ldots,c_n,\ldots: \ \forall n > N \ c_n > 0$ и ряд $\sum_{n=1}^{\infty}\frac{1}{c_n}$ -- расходится. Если
  \begin{equation*}
    \underset{n\rightarrow\infty}{\lim}(c_n \cdot \frac{a_n}{a_{n+1}} - c_{n+1}) = k,
  \end{equation*}
  то
  \begin{enumerate}
    \item При $k > 0$ ряд $(A)$ сходится;
    \item При $k < 0$ ряд $(A)$ расходится;
    \item При $k = 1$ -- ?
  \end{enumerate}
\end{theorem}

\begin{proof}
  \begin{enumerate}
    \item Пусть $k > 0$. Возьмем $0 < p < k$.

          Тогда $\exists N : \ \forall n > N$:
          \begin{multline*}
            c_n \cdot \frac{a_n}{a_{n+1}} - c_{n+1}>p \implies \\
            \implies c_n \cdot a_n - c_{n+1} \cdot a_{n+1} > p \cdot a_{n+1} > 0 \implies \\
            \implies c_n \cdot a_n > c_{n+1} \cdot a_{n+1}, \quad \forall n > N
          \end{multline*}

          Тогда последовательность $\{c_n\cdot a_n\}$ убывает и ограничена снизу $\implies$ последовательность сходится.

          Пусть $c=\underset{n\rightarrow\infty}{\lim}c_n\cdot a_n$. Рассмотрим ряд
          \begin{multline*}
            \sum_{m=1}^{n}(c_m\cdot a_m - c_{m+1}\cdot a_{m+1}) = \\
            = (c_1 \cdot a_1 - c_2 \cdot a_2) + (c_2 \cdot a_2 - c_3 \cdot a_3) + \ldots + (c_n \cdot a_n - c_{n+1}\cdot a_{n+1}) = \\
            = c_1 \cdot a_1 - c_{n+1} \cdot a_{n+1};
          \end{multline*}

          \begin{multline*}
            \underset{n\rightarrow\infty}{\lim}\sum_{m=1}^{n}(c_m \cdot a_m - c_{m+1} \cdot a_{n+1}) = \\
            = \underset{n\rightarrow\infty}{\lim}(c_1 \cdot a_1 - c_{n+1} \cdot a_{n+1}) = c_1 \cdot a_1 - c \implies
          \end{multline*}
          $\implies$ сходится ряд $\sum_{n=1}^{\infty}(c_n \cdot a_n - c_{n+1} \cdot a_{n+1}) \implies$ из того, что $c_n \cdot a_n - c_{n+1} \cdot a_{n+1} > p \cdot a_{n+1} > 0$ и 1-го признака сравнения $\implies$ ряд $\sum_{n=1}^{\infty}p\cdot a_{n+1}$ сходится $\implies$ ряд $(A)$ сходится.

    \item Пусть $k < 0 \implies \exists N: \ \forall n > N$
          \begin{multline*}
            c_n \cdot \frac{a_n}{a_{n+1}} - c_{n+1} < 0 \implies \\
            \implies \frac{a_n}{a_{n+1}} < \frac{c_{n+1}}{c_n} = \frac{\frac{1}{c_n}}{\frac{1}{c+{n+1}}} \implies \\
            \implies \frac{a_{n+1}}{a_{n}} > \frac{\frac{1}{c_{n+1}}}{\frac{1}{c_n}};
          \end{multline*}

          $\sum_{n=1}^{\infty}\frac{1}{c_n}$ расходится $\implies$ по 3-му признаку сравнения ряд $(A)$ расходится.

    \item Придумать 2 примера когда $k=0$ и ряды сходятся/расходятся.
  \end{enumerate}
\end{proof}

\begin{theorem}[признак Бертрана]
  Пусть ряд $(A) \ \sum_{n=1}^{\infty}a_n$ -- положительный. Если
  \begin{equation*}
    \underset{n\rightarrow\infty}{\lim} \ln n \cdot [n \cdot (\frac{a_n}{a_{n+1}} - 1)] = B,
  \end{equation*}
  то
  \begin{enumerate}
    \item При $B > 1$ ряд $(A)$ сходится;
    \item При $B < 1$ ряд $(A)$ расходится;
    \item При $B = 1$ ряд $(A)$ -- ?
  \end{enumerate}
\end{theorem}

\begin{proof}
  Рассмотрим ряд $\sum_{n=2}^{\infty} \frac{1}{n\cdot \ln n}$ -- расходится. Составим последовательность Кумера:
  \begin{multline*}
    k_n = \underbrace{n \cdot \ln n}_{c_n} \cdot \frac{a_n}{a_{n+1}} - \underbrace{(n+1) \cdot \ln(n+1)}_{c_{n+1}} = \\
    = \left| \ln(n+1) = \ln(n\cdot \frac{n+1}{n}) = \ln n + \ln(1 + \frac{1}{n}) \right| = \\
    = n \cdot \ln n \cdot \frac{a_n}{a_{n+1}} - (n+1)\cdot (\ln n + \ln(1 + \frac{1}{n})) = \\
    = n \cdot \ln n \cdot \frac{a_n}{a_{n+1}} - n\cdot \ln n - \ln n - \ln(1 + \frac{1}{n})^{n+1} = \\
    = \ln n (n \cdot \frac{a_n}{a_{n+1}} - n - 1) - \ln (1 + \frac{1}{n})^{n+1} = \\
    = \ln n \cdot (n(\frac{a_n}{a_{n+1}} - 1) - 1) - \ln(1+\frac{1}{n})^{n+1};
  \end{multline*}
  \begin{multline*}
    \underset{n\rightarrow\infty}{\lim} k_n = \\
    = \underset{n\rightarrow\infty}{\lim}[\underbrace{\ln n \cdot (n(\frac{a_n}{a_{n+1}} - 1) - 1)}_{B} - \ln(\underbrace{1 + \frac{1}{n}^n}_{e}) - \ln(1 + \frac{1}{n})] = \\
    = B - 1,
  \end{multline*}
  по признаку Кумера, при $B-1 > 0$ ряд $(A)$ сходится, при $B-1 < 0$ ряд $(A)$ расходится, при $B=1$ -- ?
\end{proof}

\begin{theorem}[признак Гаусса]
  Ряд $(A) \ \sum_{n=1}^{\infty} a_n, \ a_n > 0, \ \forall n \in \mathbb{N}, \ \lambda, \mu \in \mathbb{R}$. Если
  \begin{equation*}
    \frac{a_n}{a_{n+1}} = (\lambda + \frac{\mu}{n}) + O(\frac{1}{n^2}),
  \end{equation*}
  то
  \begin{enumerate}
    \item При $\lambda > 1$, ряд $(A)$ сходится;
    \item При $\lambda < 1$, ряд $(A)$ расходится;
    \item При $\lambda = 1$ и \begin{enumerate}
            \item $\mu > 1 \implies$ ряд $(A)$ сходится;
            \item $\mu \leqslant 1 \implies$ ряд $(A)$ расходится.
          \end{enumerate}
  \end{enumerate}
\end{theorem}

\begin{proof}
  \begin{enumerate}
    \item Если $\lambda < 1$, то
          \begin{multline*}
            \underset{n\rightarrow\infty}{\lim} \frac{a_{n+1}}{a_n} = [\underset{n\rightarrow\infty}{\lim}(\lambda + \frac{\mu}{n} + O(\frac{1}{n^2}))]^{-1} = \\
            = [\underset{n\rightarrow\infty}{\lim}(\lambda + \underbrace{\frac{\mu}{n}}_{\rightarrow 0} + \underbrace{\frac{1}{n^2}}_{\rightarrow 0} \cdot \Omega(\frac{1}{n^2}))]^{-1} = \frac{1}{\lambda},
          \end{multline*}
          по признаку Даламбера, если $\frac{1}{\lambda} < 1$, то есть $\lambda > 1$, ряд $(A)$ сходится.

    \item Если $\lambda > 1$, то
          \begin{multline*}
            \underset{n\rightarrow\infty}{\lim} \frac{a_{n+1}}{a_n} = [\underset{n\rightarrow\infty}{\lim}(\lambda + \frac{\mu}{n} + O(\frac{1}{n^2}))]^{-1} = \\
            = [\underset{n\rightarrow\infty}{\lim}(\lambda + \underbrace{\frac{\mu}{n}}_{\rightarrow 0} + \underbrace{\frac{1}{n^2}}_{\rightarrow 0} \cdot \Omega(\frac{1}{n^2}))]^{-1} = \frac{1}{\lambda},
          \end{multline*}
          по признаку Даламбера, если $\frac{1}{\lambda} > 1$, то есть $\lambda < 1$, ряд $(A)$ расходится.

    \item Если $\lambda = 1$
          \begin{equation*}
            \frac{a_n}{a_{n+1}} = 1 + \frac{\mu}{n} + O(\frac{1}{n^2});
          \end{equation*}
          \begin{equation*}
            n(\frac{a_n}{a_{n+1}} - 1) = \mu + n \cdot O(\frac{1}{n^2});
          \end{equation*}
          \begin{equation*}
            \underset{n\rightarrow\infty}{\lim}(n\cdot \frac{a_n}{a_{n+1}} - 1) = \underset{n\rightarrow\infty}{\lim}(\mu + \underbrace{n \cdot \frac{1}{n^2} \cdot \Omega (\frac{1}{n^2})}_{\rightarrow 0}) = \mu \implies
          \end{equation*}
          $\implies$ по признаку Реббе $\implies \left[\begin{array}{l}
              \mu > 1 \implies (A) \rightarrow\leftarrow \\
              \mu < 1 \implies (A) \leftarrow\rightarrow \\
            \end{array} \right.$

          Пусть $\mu = 1$, тогда
          \begin{multline*}
            \underset{n\rightarrow\infty}{\lim}\ln n \cdot (n \cdot (\frac{a_n}{a_{n+1}} - 1) - 1) = \\
            = \underset{n\rightarrow\infty}{\lim}\ln n \cdot (n \cdot (1 + \frac{1}{n} + O(\frac{1}{n^2}) - 1) - 1) = \\
            = \underset{n\rightarrow\infty}{\lim}\ln n \cdot (1 + n \cdot O(\frac{1}{n^2}) - 1) = \underset{n\rightarrow\infty}{\lim}\ln n \cdot n \cdot O(\frac{1}{n^2}) = \\
            = \underset{n\rightarrow\infty}{\lim}(\ln n \cdot n \cdot \frac{1}{n^2} \cdot \Omega(\frac{1}{n^2})) = \underset{n\rightarrow\infty}{\lim}\frac{\ln n}{n} \cdot \Omega(\frac{1}{n^2}) = 0.
          \end{multline*}
          В самом деле,
          \begin{equation*}
            \underset{n\rightarrow\infty}{\lim}\frac{\ln n}{n} = \underset{n\rightarrow\infty}{\lim} \frac{1}{n} \cdot \ln n = \underset{n\rightarrow\infty}{\lim}\ln n^{\frac{1}{n}} = \underset{n\rightarrow\infty}{\lim} \ln \sqrt[1]{n} = 0 \implies
          \end{equation*}
          $\implies$ по прихнаку Бертрана ряд $(A)$ расходится.
  \end{enumerate}
\end{proof}

\section{Сходимость знакопеременных рядов}

Пусть дан ряд $(A) \ \sum_{n=1}^{\infty} a_n$. Если $\exists N: \ \forall n > N \ a_n$ не меняет знак, то исследование сходимости такого ряда сводится к исследованию сходимости положительных рядов. Будем считать, что "$+$" и "$-$" бесконечно много. Такие ряды будем называть \textbf{знакопеременными}.

\begin{definition}[абсолютно сходящийся ряд]
  Ряд $(A)$ называется \textbf{абсолютно сходящимся}, если сходится ряд
  \begin{equation*}
    (A^*) \quad \sum_{n=1}^{\infty}|a_n|
  \end{equation*}
\end{definition}

\begin{statement}
  Если ряд $(A)$ абсолютно сходится, то он сходится.
\end{statement}

\begin{proof}
  $"\impliedby"$ Пусть ряд $(A)$ абсолютно сходящийся, то есть сходится ряд $(A^*) \ \sum_{n=1}^{\infty}|a_n| \implies$ по критерию Коши $\forall \epsilon > 0 \ \exists N: \ \forall n > N \ \forall p > 0$
  \begin{equation*}
    |a_{n+1}| + |a_{n+1}| + \ldots + |a_{n+1}| < \epsilon
  \end{equation*}
  $"\implies"$ Пусть $\epsilon > 0$ задано. Рассмотрим
  \begin{equation*}
    |A_{n+p} - A_n| = |a_{n+1} + \ldots + a_{n+p}| \leqslant |a_{n+1}| + \ldots + |a_{n+p}| < \epsilon
  \end{equation*}
  $\implies$ ряд $(A)$ сходится.
\end{proof}

\begin{definition}[условно сходящийся ряд]
  Если ряд $(A)$ сходится, а ряд $(A^*)$ расходится, то ряд $(A)$ называется \textbf{условно сходящимся}.
\end{definition}

\begin{definition}[знакочередующийся ряд]
  Ряд $(A)$ называется \textbf{знакочередующимся}, если $\forall n \in \mathbb{N} \quad a_n \cdot a_{n+1} < 0$. Обозначим знакочередующийся ряд:
  \begin{equation*}
    (\overline{A}) \quad \sum_{n=1}^{\infty}(-1)^{n-1}\cdot a_n, \quad a_n > 0 \ \forall n \in\mathbb{N}
  \end{equation*}
\end{definition}

\begin{theorem}[признак Лейбница]
  Пусть ряд $(\overline{A}) \ \sum_{n=1}^{\infty}(-1)^{n-1}\cdot a_n \ (a_n > 0 \ \forall n)$ удовлетворяет условиям:
  \begin{enumerate}
    \item $a_1 \geqslant a_2 \geqslant a_3 \geqslant \ldots \geqslant a_n \geqslant \ldots$;
    \item $\underset{n\rightarrow\infty}{\lim} a_n = 0$.
  \end{enumerate}

  Тогда ряд $(\overline{A})$ сходится и его сумма $S: \quad 0 < S \leqslant a_1$.
\end{theorem}

\begin{proof}
  Рассмотрим
  \begin{multline*}
    S_{2n} = a_1 - a_2 + a_3 - \ldots + a_{2n - 1} - a_{2n} = \\
    = (a_1 - a_2) + (a_3 - a_4) + \ldots + (a_{2n-1} - a_{2n}),
  \end{multline*}
  тогда $\forall i: \ a_i - a_{i+1} \geqslant 0 \implies S_{2n}\geqslant 0 \ \forall n \implies$ последовательность $S_{2n} \nearrow$.

  С другой стороны,
  \begin{equation*}
    S_{2n} = a_1 - \underbrace{(a_2 - a_3)}_{\geqslant 0} - \underbrace{(a_4 - a_5)}_{\geqslant 0} - \ldots - \underbrace{(a_{2n-2} - a_{2n-1})}_{\geqslant0} - a_{2n}
  \end{equation*}
  $\implies S_{2n} \leqslant a_1 \ \forall n$.

  Таким образом, $S_{2n}$ не убывает и ограничена сверху $\implies$ по теореме Вейерштрасса $\implies \exists \underset{n\rightarrow\infty}{\lim} S_{2n} = S$.

  Далее,
  \begin{equation*}
    \underset{n\rightarrow\infty}{\lim}S_{2n+1} = \underset{n\rightarrow\infty}{\lim}(S_{2n} + a_{2n+1}) = \underset{n\rightarrow\infty}{\lim}S_{2n} + \underset{n\rightarrow\infty}{\lim} a_{2n+1} = S + 0 = S.
  \end{equation*}

  Таким образом, $\underset{n\rightarrow\infty}{\lim}S_n = S$.

  Так как $0 < S_n \leqslant a_1$ (если $S_n = 0$, то $a_1$ может быть $=0$, что невозможно, так как $a_1 > 0$) $\implies$ (берем пределы от неравенства) $0 < S \leqslant a_1$.
\end{proof}

\begin{effect}
  Если знакочередующийся ряд $(\overline{A})$ сходится, то сумма его $n$-го остатка имеет знак $(n+1)$-го члена ряда и не больше его по модулю.
\end{effect}

\begin{example}
  \begin{enumerate}
    \item Рассмотрим ряд
          \begin{equation*}
            (\overline{H}) \quad \sum_{n=1}^{\infty}(-1)^{n-1}\frac{1}{n} = 1 - \frac{1}{2} + \frac{1}{3} - \frac{1}{4} + \ldots + (-1)^{n-1}\frac{1}{n} + \ldots,
          \end{equation*}
          по признаку Лейбница:
          \begin{enumerate}
            \item $1 > \frac{1}{2} > \frac{1}{3} > \ldots > \frac{1}{n}$;
            \item $\underset{n\rightarrow\infty}{\lim}\frac{1}{n} = 0$
          \end{enumerate}
          $\implies (\overline{H})$ сходится, $0 < S \leqslant 1$;

    \item Рассмотрим $\sum_{n=1}^{\infty}|(-1)^{n-1}\frac{1}{n}| = \sum_{n=1}^{\infty}\frac{1}{n}$ -- расходится $\implies$ ряд $(\overline{H})$ -- условно сходящийся.
  \end{enumerate}
\end{example}

\begin{theorem}[признак Абеля и Дирихле]
  \begin{enumerate}
    \item (Абеля) Если \begin{itemize}
            \item последовательность $\{a_n\}$ монотонна и ограничена;
            \item ряд $\sum_{n=1}^{\infty} b_n$ сходится, то
          \end{itemize}
          ряд $\sum_{n=1}^{\infty}a_n \cdot b_n$ сходится.

    \item (Дирихле) Если \begin{itemize}
            \item последовательность $\{a_n\}$ монотонна и $\underset{n\rightarrow\infty}{\lim}a_n = 0$;
            \item частичные суммы ряд $(B) \ \sum_{n=1}^{\infty}b_n$ ограничены, то есть $\exists k > 0: \ \forall n \quad |\sum_{m=1}^{n} b_m < k$, то $\sum_{n=1}^{\infty}a_n \cdot b_n$ сходится.
          \end{itemize}
  \end{enumerate}
\end{theorem}

\begin{proof}[Доказательство теоремы]
  \begin{lemma}
    \begin{enumerate}
      \item Если числа $a_1,a_2,\ldots,a_n$ либо не возрастают, либо не убывают;
      \item Суммы $B_1 = b_1, \ B_2 = b_1 + b_2, \ \ldots, \ B_n = b_1 + b_2 + \ldots + b_n: \ \forall k = 1,\ldots,n \quad |B_k| \leqslant L$, тогда
            \begin{equation}
              \quad \bigg|\sum_{k=1}^{n} a_k \cdot b_k \bigg| \leqslant L\cdot (|a_1| + |a_n|)
            \end{equation}
    \end{enumerate}
  \end{lemma}

  \begin{proof}[Доказательство леммы]
    Рассмотрим
    \begin{multline*}
      a_1 \cdot b_1 + a_2 \cdot b_2 + \ldots + a_n \cdot b_n = \\
      = a_1 \cdot B_1 + a_2 \cdot (B_2 - B_1) + a_3 \cdot (B_3 - B_2) + \ldots + a_n \cdot (B_n - B_{n-1}) = \\
      = a_1 \cdot B_1 + a_2 \cdot B_2 - a_2 \cdot B_1 + a_3\cdot B_3 - a_3 \cdot B_2 + \ldots + a_n \cdot B_n - a_n \cdot B_{n-1} = \\
      = B_1\cdot (a_1 - a_2) + B_2\cdot (a_2 - a_3) + B_3 \cdot (a_3 - a_4) + \ldots + B_{n-1} \cdot (a_{n-1} - a_n) + a_n \cdot B_n = \\
      = \sum_{k=1}^{n-1} B_k \cdot (a_k - a_{k-1}) + a_n \cdot B_n.
    \end{multline*}

    Таким образом,
    \begin{multline*}
      \bigg|\sum_{k=1}^{n}a_k\cdot b_k \bigg| = \bigg|\sum_{k=1}^{n-1} B_k \cdot (a_k - a_{k+1}) + a_n \cdot B_n\bigg| \leqslant \\
      \leqslant \sum_{k=1}^{n-1}|B_k| \cdot |a_k - a_{k+1} + |a_n| \cdot |B_n| \leqslant L\cdot \bigg(\sum_{k=1}^{n-1}|a_k - a_{k+1}| + |a_n|\bigg) = \\
      = L\cdot (|a_1| + |a_n| + |a_n|) = L\cdot (|a_1| + 2\cdot |a_n|).
    \end{multline*}
  \end{proof}

  \begin{enumerate}
    \item Пусть выполнены условия признака Абеля. Тогда $\exists M > 0: \ |a_n| \leqslant M$. Пусть $\epsilon > 0$ задано. Возьмем номер $N: \ \forall n > N, \ \forall p > 0$
          \begin{equation*}
            \bigg|\sum_{k=n+1}^{n+p}b_k\bigg| < \epsilon^* = \frac{\epsilon}{3\cdot M}.
          \end{equation*}

          Частичные суммы ряда $\sum_{n=1}^{\infty}a_n\cdot b_n$ имеют вид $S_n = a_1\cdot b_1 + \ldots + a_n \cdot b_n$. По критерию Коши найдем $N_1: \ \forall n > N_1, \forall p > 0$
          \begin{equation*}
            |S_{n+p} - S_n| < \epsilon,
          \end{equation*}
          \begin{multline*}
            |a_{n+1} \cdot b_{n+1} + a_{n+2} \cdot b_{n+2} + \ldots + a_{n+p} \cdot b_{n+p}| \leqslant \\
            \leqslant \epsilon^* \cdot (|a_{n+1}| + 2 \cdot |a_{n+p}|) \leqslant \epsilon^* \cdot 3 \cdot M = \frac{\epsilon}{3 \cdot M} = \epsilon \implies
          \end{multline*}
          $\implies$ по критерию Коши ряд $\sum_{n=1}^{\infty}a_n \cdot b_n$ сходится.

    \item Пусть выполнены условия признака Дирихле. Так как $\underset{n\rightarrow\infty}{\lim}a_n = 0$, то $\exists N: \ \forall n > N \quad (\epsilon > 0$ задано):
          \begin{equation*}
            |a_n| < \frac{\epsilon}{3 \cdot k}, \quad \bigg|\sum_{k=1}^{n}b_k\bigg| \leqslant k.
          \end{equation*}

          По критерию Коши:
          \begin{multline*}
            |S_{n+p} - S_n| = |a_{n+1} \cdot b_{n+1} + \ldots + a_{n+p} \cdot b_{n+p}| \overset{by \ lemma}{\leqslant} \\
            \leqslant k\cdot(|a_{n+1}| + 2\cdot |a_{n+p}|) < k\cdot \frac{3\cdot \epsilon}{3 \cdot k} = \epsilon.
          \end{multline*}
  \end{enumerate}
\end{proof}

\begin{example}
  $\sum_{n=1}^{\infty} \frac{\sin (n\cdot x)}{n} = \sum_{n=1}^{\infty} \frac{1}{n} \cdot \sin (n\cdot x)$ \\

  $a_n = \frac{1}{n}\rightarrow 0$ при $n\rightarrow\infty$. Оценим частичную сумму $\sum_{n=1}^{\infty}\sin (n\cdot x)$:
  \begin{multline*}
    \sin x + \sin (2\cdot x) + \sin (3\cdot x) + \ldots + \sin (n\cdot x) = \\
    = \frac{1}{\sin \frac{x}{2}} \cdot \bigg(\sin x \cdot \sin \frac{x}{2} + \sin (2\cdot x) \cdot \sin\frac{x}{2}\bigg) = \\
    = \frac{1}{2} \cdot \frac{1}{\sin\frac{x}{2}} \cdot \bigg(\cos \frac{x}{2} - \cos \frac{3\cdot x}{2} + \cdot \frac{3\cdot x}{2} - \cos \frac{5\cdot x}{2} + \ldots \\
    \ldots + \cos \frac{(2\cdot n-1)\cdot x}{2} - \cos \frac{(2\cdot n+1) \cdot x}{2}\bigg) = \\
    = \frac{1}{2} \cdot \frac{1}{\sin \frac{x}{2}} \cdot \bigg(\cos \frac{x}{2} - \cos \frac{(2\cdot n + 1) \cdot x}{2}\bigg) = \\
    = \frac{2\cdot \sin \frac{(n+1)\cdot x}{2} \cdot \sin \frac{n\cdot x}{2}}{2\cdot \sin \frac{x}{2}}.
  \end{multline*}

  Тогда
  \begin{equation*}
    \bigg|\sum_{k=1}^{n} \sin (k\cdot x)\bigg| = \bigg|\frac{\sin\frac{(n+1)\cdot x}{2}\cdot \sin \frac{n \cdot x}{2}}{\sin \frac{x}{2}}\bigg| \leqslant \frac{1}{\sin\frac{x}{2}},
  \end{equation*}
  $\frac{x}{2} \ne \pi \cdot k, \ k \in \mathbb{Z} \implies x \ne 2 \cdot \pi \cdot k, \ k \in \mathbb{Z}$. \\

  По признаку Дирихле ряд $\sum_{n=1}^{\infty}\frac{\sin (n\cdot x)}{n}$ сходится.
\end{example}

\section{Свойства сходящихся рядов}

Рассмотрим ряд
\begin{equation*}
  1-1+1-1+1-1+\ldots
\end{equation*}

Если
\begin{equation*}
  (1-1) + (1-1) + \ldots + (1-1) + \ldots,
\end{equation*}
\begin{equation*}
  1 + (-1 + 1) + (-1 + 1) + \ldots
\end{equation*}

Пусть дан ряд
\begin{equation*}
  (A) \quad \sum_{n=1}^{\infty}a_n.
\end{equation*}

Составим из ряда $(A)$ ряд $(\widetilde{A})$:
\begin{multline*}
  \underbrace{(a_1 + a_2 + \ldots + a_{n_1})}_{\widetilde{a}_1} + \underbrace{(a_{n_1 + 1} + a_{n_1 + 2} + \ldots + a_{n_2})}_{\widetilde{a}_2} + \ldots \\
  \ldots + \underbrace{(a_{n_k + 1} + \ldots + a_{n_k + 1})}_{\widetilde{a}_{k+1}} + \ldots = (\widetilde{A}) \quad \sum_{k=1}^{\infty}\sum_{l=n_{k-1}+1}^{n_k} a_l, \quad a_{n_0} = a_1.
\end{multline*}

\begin{theorem}[сочетательное свойство сходящихся рядов]
  \begin{enumerate}
    \item Если ряд $(A)$ сходится, то для любой возрастающей последовательности $n_k$ ряд $(\widetilde{A})$ сходится и их суммы совпадают ($A = \widetilde{A}$);
    \item Если ряд $(\widetilde{A})$ сходится и внутри каждой  скобки знак не меняется, то ряд $(A)$ сходится и их суммы совпадают, то есть $\widetilde{A} = A$.
  \end{enumerate}
\end{theorem}

\begin{proof}
  \begin{enumerate}
    \item Пусть ряд $(A)$ сходится, $\widetilde{A}_k$ -- частичные суммы ряда $(\widetilde{A})$:
          \begin{equation*}
            \begin{array}{l}
              \widetilde{A}_1 = \widetilde{a}_1 = \sum_{k=1}^{n_1}a_k = A_{n_1}                         \\
              \widetilde{A}_2 = \widetilde{a}_1 + \widetilde{a}_2 = \sum_{k=n_1 + 1}^{n_2}a_k = A_{n_1} \\
              \vdots                                                                                    \\
              \widetilde{A}_k = A_{n_k}
            \end{array}
          \end{equation*}

          Так как ряд $(A)$ сходится, то существует
          \begin{equation*}
            \underset{k\rightarrow\infty}{\lim}A_{n_k} = A \implies A = \underset{k\rightarrow\infty}{\lim}A_{n_k} = \underset{n\rightarrow\infty}{\lim}\widetilde{A}_k = \widetilde{A}
          \end{equation*}

    \item Пусть ряд $(\widetilde{A})$ сходится. Имеем, при
          \begin{eqnarray*}
            a_1 > 0: \quad A_1 < A_2 < \ldots < A_{n_1} \\
            a_1 < 0: \quad A_1 > A_2 > \ldots > A_{n_1}
          \end{eqnarray*}

          \begin{itemize}
            \item Далее, если $a_{n_1 + 1} > 0$, тогда при
                  \begin{equation*}
                    a_1 > 0: \quad A_{n_1 + 1} < A_{n_1 + 2} < \ldots < A_{n_2} \implies
                  \end{equation*}
                  $\implies A_{n_1} = \widetilde{A}_1 < A_{n_2} = \widetilde{A}_2$, при $a_1 < 0 \implies A_{n_1} < 0$ и $A_{n_1} < A_{n_2} \implies \widetilde{A}_1 < \widetilde{A}_2$ если $a_{n_1 + 1} < 0$.

            \item Если же $a_{n_1 + 1} < 0$, тогда при
                  \begin{eqnarray*}
                    a_1 < 0: & \quad A_{n_1} = & \widetilde{A}_1 > A_{n_2} = \widetilde{A}_2 \\
                    a_1 > 0: & \quad A_{n_1} = & \widetilde{A}_1 > \widetilde{A}_2
                  \end{eqnarray*}
          \end{itemize}

          Аналогично, пока $n$ меняется от $n_k$ до $n_{k+1}$, то будем иметь либо $A_{n_k} < A_n < A_{n_{k+1}}$, либо $A_{n_k} > A_n > A_{n_{k+1}}$.

          Ряд $(\widetilde{A})$ -- сходится $\implies \exists \underset{k\rightarrow\infty}{\lim}\widetilde{A}_k = \underset{k\rightarrow\infty}{\lim}\widetilde{A}_{k+1} = \widetilde{A} \implies$ по теореме о 2-х миллиционерах:
          \begin{equation*}
            \underset{k\rightarrow\infty}{\lim}A_n = \widetilde{A}
          \end{equation*}
  \end{enumerate}
\end{proof}

\begin{theorem}[переместительное свойство сходящихся рядов]
  Если ряд $(A) \quad \sum_{n=1}^{\infty}a_n$ абсолютно сходится, то его сумма не зависит от перестановки членов ряда.
\end{theorem}

\begin{proof}[Доказательство теоремы]
  Пусть ряд
  \begin{equation*}
    (A) \quad \sum_{n=1}^{\infty} a_n
  \end{equation*}
  сходится абсолютно $\implies$ ряд
  \begin{equation*}
    (A^*) \quad \sum_{n=1}^{\infty}|a_n|
  \end{equation*}
  сходится. Пусть ряд
  \begin{equation*}
    (A') \quad \sum_{n=1}^{\infty}a_n'
  \end{equation*}
  получен из ряда $(A)$ путем перестановки его членов. Покажем, что ряд $(A')$ сходится и $A = A'$ (их суммы совпадают).

  \begin{enumerate}
    \item Пусть $(A)$ -- знакоположительный, то есть $\forall n \in \mathbb{N} \quad a_n > 0$. Рассмотрим частичные суммы ряда $(A')$:
          \begin{equation*}
            A_k' = a_1' + a_2' + \ldots + a_k' = a_{n_1} + a_{n_2} + \ldots + a_{n_k}.
          \end{equation*}

          Пусть $n' = \max\{n_1,n_2,\ldots,n_k\}$. Тогда
          \begin{equation*}
            A_k' \leqslant a_1 + a_2 + \ldots + a_{n_j} + \ldots + a_{n'} = A_{n'},
          \end{equation*}
          где $A_{n'}$ -- $n'$-я частичная сумма ряда $(A)$. Так как $(A)$ сходится и знакоположительный $\implies A_{n'} \leqslant A$.

          Таким образом получаем, что $\forall k \quad A_k' \leqslant A \implies$ последовательность $A_k' \nearrow$ и ограничена $\implies$
          \begin{equation*}
            \exists\underset{k\rightarrow\infty}{\lim}A_k' = A' \leqslant A.
          \end{equation*}

          С другой стороны, ряд $(A')$ получен перестановкой членов ряда $(A) \implies A' \geqslant A \implies A' \leqslant A \leqslant A' \implies A = A'$.

    \item Пусть ряд $(A)$ сходится абсолютно, то есть $(A^*) \quad \sum_{n=1}^{\infty} |a_n|$ сходится. С рядом $(A)$ свяжем два ряда:
          \begin{equation*}
            (P) \ \sum_{n=1}^{\infty}p_n, \quad (Q) \ \sum_{n=1}^{\infty}q_n,
          \end{equation*}
          где $p_n$ -- положительные члены ряда $(A)$, $q_n$ -- отрицательные члены ряда $(A)$, взятые по модулю, причем все члены рядов $(P)$ и $(Q)$ взяты в том же порядке, как они стояли в ряде $(A)$.
  \end{enumerate}

  \begin{lemma}
    Если ряд $(A)$ абсолютно сходящийся, то ряды $(P)$ и $(Q)$ сходятся и $A = P - Q$.
  \end{lemma}

  \begin{proof}[Доказательство леммы]
    Пусть $(A^*) \quad \sum_{n=1}^{\infty}|a_n|$ -- сходится $\implies \sum_{n=1}^{\infty}|a_n| = A^*$. $A_n^*$ -- частичные суммы ряда $(A^*)$. Имеем $P_{n_k} = a_{n_1} + a_{n_2} + \ldots + a_{n_k}$, где $n_1 < n_2 < \ldots < n_k \leqslant n, \quad P_{n_k} \leqslant A_n^*$,
    \begin{eqnarray*}
      & (A) & \ \sum_{n=1}^{\infty} a_n = a_1 + a_2 + a_3 + a_4 + a_5 + a_6 \\
      & (P) & \ \sum_{n=1}^{\infty} p_n = \underbrace{a_1 + a_3 + a_4 + a_6}_{P_3} \\
      & (A^*) & \ \underbrace{|a_1| + |a_2| + |a_3| + |a_4| + |a_5|}_{A_5^*} + \ldots
    \end{eqnarray*}
    и $Q_{n_m} \leqslant A_n^* \implies$ (так как $(A)^*$ сходится) $\implies A_n^* \leqslant A^* \leftarrow$ сумма ряда $(A^*)$ и $\implies P_{n_k}\leqslant A^*$ и $Q_{n_m} \leqslant A^*$.

    Далее, $A_n = P_{n_k} - Q_{n_m}$, где $n_k \leqslant n, \ n_m \leqslant n$. При $n\rightarrow\infty \implies k \rightarrow\infty, \ m \rightarrow\infty$.

    Далее, так как $(A)$ сходится абсолютно $\implies (A)$ сходится $\implies$
    \begin{multline*}
      \implies \exists A = \underset{n\rightarrow\infty}{\lim}A_n = \underset{\underset{k\rightarrow\infty}{m\rightarrow\infty}}{\lim}(P_{n_k} - Q_{n_m}) = \\
      = \underset{k\rightarrow\infty}{\lim}P_{n_k} - \underset{m\rightarrow\infty}{\lim}Q_{n_m} = P - Q.
    \end{multline*}
  \end{proof}

  Если ряд $(A)$ сходится абсолютно, то сходится ряд $(A^*)$, $(A^*)$ -- положительный ряд $\implies (A^{*'})$ сходится (получен путем перестановки членов ряда $(A^*)$) $\implies$ по лемме сходятся ряды $(P')$ и $(Q')$ \\

  {\Large ТУТ ХИМИЧЕСКИЕ ФОРМУЛЫ} \\

  и $A' = P' - Q'$.

  \begin{itemize}
    \item $P'$ -- положительный ряд $\implies$ по пункту 1, $P$ -- сходится;
    \item $Q'$ -- положительный ряд $\implies$ по пункту 1, $Q$ -- сходится
  \end{itemize}
  и $P' = P, \quad Q' = Q \implies A' = P - Q = A$.
\end{proof}

\begin{theorem}[Римана о перестановке членов условно сходящегося ряда]
  Если ряд $(A)$ условно сходится, то $\forall B \in \mathbb{R}$ (в том числе $B = \pm\infty$) $\exists$ перестановка ряда $(A)$ такая, что полученный ряд сходится и имеет сумму $B$. Более того, $\exists$ перестановка ряда $(A)$ такая, что частичные суммы полученного ряда не стремятся ни к конечному, ни к бесконечному пределу.
\end{theorem}

\begin{proof}[Доказательство теоремы]
  \begin{lemma}
    Если ряд $(A)$ сходится условно, то ряды $(P)$ и $(Q)$ расходятся.
  \end{lemma}

  \begin{proof}[Доказательство леммы]
    Рассмотрим $A_n = P_k - Q_m$, где $k \leqslant n, \ m \leqslant n \ (k + m = n)$
    \begin{eqnarray*}
      A_n^* = P_k^* + Q_m^* \quad (k + m = n), \\
      \underset{n\rightarrow\infty}{\lim} A_n = A; \quad \underset{n\rightarrow\infty}{\lim}A_n^* = \infty.
    \end{eqnarray*}

    Допустим, что ряд $(P)$ сходится $\implies (P^*)$ сходится, а так же $\exists \underset{k\rightarrow\infty}{\lim}P_k = P \implies \exists\underset{m\rightarrow\infty}{\lim}Q_m = A - P \overset{from \ (*)}{\implies} Q^*$ -- сходится $\implies A^*$ имеет предел. Противоречие $\implies P$ расходится.

    Для $Q$ -- аналогично.
  \end{proof}

  Пусть $B \in \mathbb{R}$. Возьмем номер $n_1: \ p_1 + p_2 + \ldots + p_{n_1} \geqslant B$. Выберем $n_2: \ p_1 + p_2 + \ldots + p_{n_1} - q_1 - q_2 - \ldots - q_{n_2} \leqslant B$.

  Более того, элементы $p$ и $q$ будем брать столько, сколько это необходимо для выполнения этого условия.

  Возьмем $n_3: \ p_1 + p_2 + \ldots + p_{n_1} - q_1 - q_2 - \ldots - q_{n_2} + p_{n_1 + 1} + p_{n_1 + 2} + \ldots + p_{n_3} \geqslant B$ и так далее.

  Таким образом получим ряд $(p_1 + \ldots + p_{n_1}) + (-q_1 - \ldots - q_{n_2}) + (p_{n_1 + 1} + \ldots + p_{n_3}) + (-q_{n_2 + 1} - \ldots - q_{n_4}) + \ldots$ -- этот ряд сходится к $B$.

  Действительно, так как ряд $(A)$ сходится, то $\underset{n\rightarrow\infty}{\lim} a_n = 0$.

  Так как количество членов $p_i$ и $q_i$ бралось лишь столько, сколько необходимо, то соответствующие частичные суммы отличаются от $B$ разве что на последнее слогаемое в этой частичной сумме, которое стремится к нулю $\implies \underset{n\rightarrow\infty}{\lim}A_n' = B$.
\end{proof}

\section{Умножение рядов}

Пусть даны ряды
\begin{equation*}
  (A) \ \sum_{n=1}^{\infty}a_n, \quad (B) \ \sum_{n=1}^{\infty}b_n.
\end{equation*}

Составим таблицу:
\begin{center}
  \begin{tabular}{c | c | c | c | c | c}
             & $a_1$     & $a_2$      & $\cdots$ & $a_n$     & $\cdots$ \\
    \hline
    $b_1$    & $a_1 b_1$ & $a_2 b_1 $ & $\cdots$ & $a_n b_1$ & $\cdots$ \\
    \hline
    $b_2$    & $a_1 b_2$ & $a_2 b_2$  & $\cdots$ & $a_n b_2$ & $\cdots$ \\
    \hline
    $\vdots$ & $\vdots$  & $\vdots$   & $\ddots$ & $\cdots$  & $\cdots$ \\
    \hline
    $b_n$    & $a_1 b_n$ & $a_2 b_n$  & $\vdots$ & $a_n b_n$ & $\cdots$ \\
    \hline
    $\vdots$ & $\vdots$  & $\vdots$   & $\vdots$ & $\vdots$  & $\ddots$ \\
  \end{tabular}
\end{center}

\begin{definition}[произведение рядов, форма Коши]
  \textbf{Произведением рядов} $(A)$ и $(B)$ назовем ряд, членами которого ялвяются элементы на строке таблицы $a_ib_j$, взятые в произвольном порядке.

  Если числа выбираются по диагоналям, то произведение называется \textbf{формой Коши}:
  \begin{equation*}
    a_1 b_1 + (a_1 b_2 + a_2 b_1) + \ldots
  \end{equation*}
\end{definition}

\begin{theorem}[Коши о произведении рядов]
  Если ряды
  \begin{equation*}
    (A) \ \sum_{n=1}^{\infty}a_n, \quad (B) \ \sum_{n=1}^{\infty}b_n
  \end{equation*}
  абсолютно сходятся, $A$ и $B$ -- их суммы, то $\forall$ их произведение абсолютно сходится и равно $A \cdot B$.
\end{theorem}

\begin{proof}
  Рассмотрим $r$-тую частичную сумму ряда
  \begin{equation*}
    (A\cdot B)^* \quad \sum_{r=1}^{\infty}|a_{n_r}\cdot b_{k_r}|,
  \end{equation*}
  \begin{multline*}
    S_r = |a_{n_1} \cdot b_{k_1}| + |a_{n_2} \cdot b_{k_2}| + \ldots + |a_{n_r} \cdot b_{k_r}| \leqslant \\
    \leqslant (|a_{n_1}| + |a_{n_2}| + \ldots + |a_{n_r}|) \cdot (|b_{k_1}| + |b_{k_2}| + \ldots + |b_{k_r}|) \leqslant \\
    \leqslant (|a_1| + |a_2| + \ldots + |a_m|) \cdot (|b_1| + |b_2| + \ldots + |b_m|),
  \end{multline*}
  где $m = \max\{n_1,n_2,\ldots,n_r,k_1,k_2,\ldots,k_r\}$.

  Так как ряды $(A)$ и $(B)$ сходятся абсолютно, то есть сходятся ряды $(A^*)$ и $(B^*)$, то $S_r \leqslant A^* \cdot B^* \implies$ последовательность $S_r \nearrow$ и ограничена $\implies \exists \underset{r\rightarrow\infty}{\lim} S_r \implies$ ряд $(A\cdot B)^*$ сходится $\implies$ ряд
  \begin{equation*}
    (A\cdot B) \quad \sum_{r=1}^{\infty}a_{n_r}\cdot b_{k_r}
  \end{equation*} -- сходится, причем его сумма не зависит от порядка суммирования.

  Будем суммировать ряд $A\cdot B$ по квадратам
  \begin{equation*}
    \underbrace{a_1b_1}_{c_1} + \underbrace{(a_1b_2 + a_2 b_2 + a_2 b_1)}_{c_2} + \underbrace{(a_1b_3 + a_2b_3 + a_3b_3 + a_3b_2 + b_3b_1)}_{c_3} + \ldots
  \end{equation*}
  \begin{equation*}
    \begin{array}{l}
      S_1 = a_1b_1 = A_1\cdot B_1                                                                        \\
      S_2 = c_1 + c_2 = a_1b_1 + (a_1b_2 + a_2b_2 + a_2b_1) = (a_1 + a_2)\cdot(b_1 + b_2) = A_2\cdot B_2 \\
      S_3 = c_1 + c_2 + c_3 = (a_1 + a_2 + a_3)\cdot(b_1 + b_2 + b_3) = A_3\cdot b_3                     \\
      \vdots                                                                                             \\
      S_n = A_n \cdot B_n
    \end{array}
  \end{equation*}
  \begin{equation*}
    \underset{n\rightarrow\infty}{\lim}S_n = \underset{n\rightarrow\infty}{\lim}(A_n \cdot B_n) = \underset{n\rightarrow\infty}{\lim}A_n \cdot \underset{n\rightarrow\infty}{\lim}B_n = A\cdot B
  \end{equation*}
\end{proof}

\section{Двойные и повторные ряды}

Рассмотрим таблицу $(\star)$
\begin{center}
  \begin{tabular}{c | c | c | c | c | c}
    $a_{11}$ & $a_{12}$ & $a_{13}$ & $\cdots$ & $a_{1k}$ & $\cdots$ \\
    \hline
    $a_{21}$ & $a_{22}$ & $a_{23}$ & $\cdots$ & $a_{2k}$ & $\cdots$ \\
    \hline
    $\vdots$ & $\vdots$ & $\vdots$ & $\ddots$ & $\cdots$ & $\cdots$ \\
    \hline
    $a_{n1}$ & $a_{n2}$ & $a_{n3}$ & $\vdots$ & $a_{nk}$ & $\cdots$ \\
    \hline
    $\vdots$ & $\vdots$ & $\vdots$ & $\vdots$ & $\vdots$ & $\ddots$ \\
  \end{tabular}
\end{center}

\begin{definition}[повторный ряд]
  \textbf{Повторным рядом} называется выражение
  \begin{equation}
    \sum_{n=1}^{\infty}\sum_{k=1}^{\infty}a_{nk},
  \end{equation}
  \begin{equation}
    \sum_{k=1}^{\infty}\sum_{n=1}^{\infty}a_{nk}.
  \end{equation}

  Говорят, что ряд $(2.7)$ сходится, если сходятся все ряды $(A_n)$ по строкам $(\sum_{k=1}^{\infty}a_{n_k} = A_n)$ и сходится ряд
  \begin{equation*}
    \sum_{n=1}^{\infty}A_n.
  \end{equation*}
\end{definition}

\begin{definition}
  \textbf{Двойным рядом} называется выражение
  \begin{equation}
    \sum_{n,k = 1}^{\infty} a_{nk}
  \end{equation}

  Говорят, что ряд $(2.9)$ сходится, если $\exists$
  \begin{equation*}
    A = \underset{\begin{array}{c}
        N\rightarrow\infty \\
        K\rightarrow\infty
      \end{array}}{\lim} = \underset{\begin{array}{c}
        N\rightarrow\infty \\
        K\rightarrow\infty
      \end{array}}{\lim}\sum_{n=1}^{N}\sum_{k=1}^{K}a_{nk}.
  \end{equation*}

  То есть $\forall \epsilon > 0 \ \exists N_0$ и $K_0: \ \forall N > N_0$ и $\forall k > K_0$
  \begin{equation*}
    \bigg|\underbrace{\sum_{n=1}^{N}\sum_{k=1}^{K}a_{nk}}_{A_{NK}} - A\bigg| < \epsilon
  \end{equation*}
\end{definition}

\begin{definition}[простой ряд]
  Пусть ряд
  \begin{equation}
    (U) \quad \sum_{r=1}^{\infty}U_r
  \end{equation}
  построен из элементов таблицы, взятых в произвольном порядке. Такой ряд будем называть \textbf{простым}, связанным с данной таблицей.
\end{definition}

\begin{theorem}["Главная"]
  Пусть дана таблица
  \begin{center}
    \begin{tabular}{c | c | c | c | c | c}
      $a_{11}$ & $a_{12}$ & $a_{13}$ & $\cdots$ & $a_{1k}$ & $\cdots$ \\
      \hline
      $a_{21}$ & $a_{22}$ & $a_{23}$ & $\cdots$ & $a_{2k}$ & $\cdots$ \\
      \hline
      $\vdots$ & $\vdots$ & $\vdots$ & $\ddots$ & $\cdots$ & $\cdots$ \\
      \hline
      $a_{n1}$ & $a_{n2}$ & $a_{n3}$ & $\vdots$ & $a_{nk}$ & $\cdots$ \\
      \hline
      $\vdots$ & $\vdots$ & $\vdots$ & $\vdots$ & $\vdots$ & $\ddots$ \\
    \end{tabular}
  \end{center}
  и по ней построены ряды
  \begin{equation*}
    \sum_{n=1}^{\infty}\sum_{k=1}^{\infty}a_{nk}, \quad \sum_{k=1}^{\infty}\sum_{n=1}^{\infty}a_{nk}, \quad \sum_{n,k = 1}^{\infty} a_{nk}, \quad \sum_{r=1}^{\infty}U_r.
  \end{equation*}

  Если после замены элементов таблицы их модулями хотя бы один из 4-х рядов становится сходящимся, то сходятся остальные и их суммы равны.
\end{theorem}

\begin{theorem}[о связи сходимости простого и повторного рядов]
  \begin{enumerate}
    \item Если ряд
          \begin{equation*}
            \sum_{r=1}^{\infty}U_r
          \end{equation*}
          абсолютно сходится, то ряд
          \begin{equation*}
            \sum_{n=1}^{\infty}\sum_{k=1}^{\infty}a_{nk}
          \end{equation*}
          сходится и его сумма равна $U$.

    \item Если после замены элементов таблицы $(\star)$ их модулями ряд
          \begin{equation*}
            \sum_{n=1}^{\infty}\sum_{k=1}^{\infty}|a_{nk}|
          \end{equation*}
          сходится, то ряд
          \begin{equation*}
            \sum_{r=1}^{\infty}U_r
          \end{equation*}
          сходится абсолютно и суммы рядов
          \begin{equation*}
            \sum_{n=1}^{\infty}\sum_{k=1}^{\infty}a_{nk} \quad \& \quad \sum_{r=1}^{\infty}U_r
          \end{equation*}
          совпадают.
  \end{enumerate}
\end{theorem}

\begin{proof}
  \begin{enumerate}
    \item Пусть $(U^*)$ сходится. Покажем, что все ряды по строкам сходятся
          \begin{equation*}
            (A_n) \quad \sum_{k=1}^{\infty}a_{nk} \quad (\forall n \in \mathbb{N})
          \end{equation*}
          и сходится ряд
          \begin{equation*}
            \sum_{n=1}^{\infty}A_n.
          \end{equation*}

          Рассмотрим
          \begin{equation*}
            |a_{n1}| + |a_{n2}| + \ldots + |a_{nk}| \leqslant |u_1| + |u_2| + \ldots + |u_r|,
          \end{equation*}
          где $r$ выбран таким образом, чтобы среди $|u_i|$ были все слагаемые $|a_{n1}, \ldots, a_{nk}|$.

          Таким образом,
          \begin{equation*}
            \underbrace{|a_{n1}| + \ldots + |a_{nk}|}_{A_{nk}^*} \leqslant U^* \implies \exists \underset{k\rightarrow\infty}{\lim}A_{nk}^* = A_n^* \implies
          \end{equation*}
          $\implies$ ряд $\sum_{k=1}^{\infty}a_{nk} \ \forall n \in \mathbb{N}$ сходится абсолютно $\implies$ он сходится.

          Далее, пусть $\epsilon > 0$ задано. Выберем номер $r_0: \ \forall r > r_0$
          \begin{equation*}
            \sum_{i=1}^{\infty}|u_{r+i}| < \frac{\epsilon}{3}.
          \end{equation*}

          Тогда
          \begin{equation*}
            \bigg|\sum_{i=1}^{r} u_i - U\bigg| = \bigg| \sum_{i=1}^{\infty}u_{r + i}\bigg| \leqslant \sum_{i=1}^{\infty}|u_{r + i}| < \frac{\epsilon}{3}
          \end{equation*}

          Так как ряды по строкам сходятся, то $\forall n$ выберем $m(n)$:
          \begin{equation*}
            \bigg|\sum_{k=1}^{m(n)}a_{n_k} - A_n\bigg| < \frac{\epsilon}{3 \cdot 2^n}.
          \end{equation*}

          Наконец, выберем номер $N_0$ такой, что все числа $u_1, u_2,\ldots, u_{r_0}$ содержались бы в первых $N_0$ строчках.

          \begin{multline*}
            \bigg|\sum_{n=1}^{N}A_n - U\bigg| = \\
            = \bigg|\sum_{n=1}^{N_0}A_n - \sum_{n=1}^{N_0}\sum_{k=1}^{m(n)}a_{n_k} + \sum_{n=1}^{N_0}\sum_{k=1}^{m(n)}a_{n_k} - \sum_{i=1}^{r_0}u_i + \sum_{i=1}^{r_0}u_i - U\bigg| \leqslant \\
            \leqslant \sum_{n=1}^{N_0}\bigg|A_n - \sum_{k=1}^{m(n)}a_{n_k}\bigg| + \bigg|\sum_{n=1}^{N_0}\sum_{k=1}^{m(n)}a_{n_k} - \sum_{i=1}^{r_0}u_i\bigg| + \underbrace{\bigg|\sum_{i=1}^{r_0}u_i - U\bigg|}_{<\frac{\epsilon}{3}} < \\
            < \frac{\epsilon}{3} + \sum_{i=r_0 + 1}^{\infty}(u_i) + \frac{\epsilon}{3} < \frac{\epsilon}{3} \cdot 3 = \epsilon.
          \end{multline*}

    \item Пусть ряд $\sum_{n=1}^{\infty}\sum_{k=1}^{\infty}|a_{n_k}| = A^*$ сходится.

          Тогда $\forall r \ \exists N,K$ такие, что числа $u_1,\ldots,u_r$ содержатся в $N$ первых строчках и $K$ первых столбцах таблицы
          \begin{equation*}
            \sum_{i=1}^{r}|u_i| \leqslant \sum_{n=1}^{N}\sum_{k=1}^{K}|a_{n_k}| \leqslant A^* \implies
          \end{equation*}
          $\implies |u_r|\nearrow$ и ограничен $\implies$ ряд $(U)$ сходится абсолютно $\implies$ по пункту 1., суммы рядов $(U)$ и $(???)$ равны.
  \end{enumerate}
\end{proof}

\begin{theorem}[свойства двойных рядов]
  \begin{enumerate}
    \item Если ряд
          \begin{equation*}
            \sum_{n,k=1}^{\infty}a_{nk}
          \end{equation*}
          сходится, то
          \begin{equation*}
            \underset{k\rightarrow\infty}{\underset{n\rightarrow\infty}{\lim}}a_{nk} = 0
          \end{equation*}

    \item (Критерий Коши)
          Ряд
          \begin{equation*}
            \sum_{n,k=1}^{\infty}a_{nk}
          \end{equation*}
          сходится тогда и только тогда, когда $\forall \epsilon > 0 \ \exists N_0,K_0: \ \forall n > N_0, \ \forall k > K_0, \ \forall p > 0, \ \forall q > 0$
          \begin{equation*}
            \bigg|\sum_{n=1}^{p}\sum_{k=1}^{q}a_{(N_0 + n)(K_0 + k)}\bigg| < \epsilon.
          \end{equation*}

    \item Если ряд
          \begin{equation*}
            \sum_{n,k=1}^{\infty}a_{nk}
          \end{equation*}
          сходится, то $\forall c \in \mathbb{R}$ ряд
          \begin{equation*}
            \sum_{n,k=1}^{\infty}(c\cdot a_{nk})
          \end{equation*}
          сходится, и его сумма равна $c\cdot A$ (где $A = \sum_{n,k=1}^{\infty}a_{nk}$).

    \item Если ряд
          \begin{equation*}
            \sum_{n,k=1}^{\infty}a_{nk}
          \end{equation*}
          сходится и ряд
          \begin{equation*}
            \sum_{n,k=1}^{\infty}b_{nk}
          \end{equation*}
          сходится, то
          \begin{equation*}
            \sum_{n,k=1}^{\infty}(a_{nk} + b_{nk}) = A + B,
          \end{equation*}
          а к тому же -- сходится.

    \item Если $\forall n, \ \forall k \ a_{nk} \geqslant 0$, то ряд
          \begin{equation*}
            \sum_{n,k=1}^{\infty}a_{nk}
          \end{equation*}
          сходится тогда и только тогда, когда его частичные суммы ограничены в совокупности.
  \end{enumerate}
\end{theorem}

\begin{proof}
  \begin{enumerate}
    \item Пусть ряд
          \begin{equation*}
            \sum_{n,k=1}^{\infty}a_{nk}
          \end{equation*}
          сходится. Заметим, что
          \begin{equation*}
            A_{nk} = \sum_{i,j=1}^{n,k},
          \end{equation*}
          \begin{equation*}
            a_{nk} = A_{nk} - A_{n(k-1)} - A_{(n-k)k} + A_{(n-1)(k-1)} \implies a_{nk} \rightarrow 0.
          \end{equation*}

    \item (Критерий Коши)
          На декартовом произведении $\mathbb{N}\times\mathbb{N}$ введем базу:
          \begin{equation*}
            B_{nk} = \{(n,k): \ n > N_0, \ k > K_0\}.
          \end{equation*}

          Тогда критерий Коши сходимости ряда -- это есть критерий Коши существования предела функции $A_{nk}$ по данной базе.

    \item Самостоятельно.

    \item Самостоятельно.

    \item $"\rightarrow"$ Очевидно.

          $"\leftarrow"$ Пусть множество $\{A_{nk}\}$ -- ограничено. Пусть $A = \sup\{A_{nk}\}$. Покажем, что $A$ -- сумма ряда
          \begin{equation*}
            \sum_{n,k=1}^{\infty}a_{nk}.
          \end{equation*}

          Пусть $\epsilon > 0$ задано. Выберем $N_0$ и $K_0$:
          \begin{equation*}
            A - A_{N_0 K_0} < \epsilon
          \end{equation*}

          ДОПИСАТЬ!!!
  \end{enumerate}
\end{proof}

\begin{theorem}[о связи сходимости двойного и простого рядов]
  Если ряд $\sum_{n,k=1}^{\infty}|a_{nk}|$ сходится, то сходится ряд $\sum_{r=1}^{\infty}u_r$.

  И наоборот, если сходится ряд $\sum_{r=1}^{\infty}|u_r|$, то сходится ряд $\sum_{n,k=1}^{\infty}a_{nk}$.

  И в обоих случаях суммы рядов равны:
  \begin{equation*}
    \sum_{n,k=1}^{\infty}a_{nk} = \sum_{r=1}^{\infty}u_r
  \end{equation*}
\end{theorem}

\begin{proof}
  $"\implies"$ Пусть двойной ряд сходится абсолютно, то есть сходится ряд $\sum_{n,k=1}^{\infty}|a_{nk}|$.

  Тогда для любого номера $S \ \exists N,K$ такие, что все числа $u_1,\ldots,u_S$ содержатся в первых $N$ строках и первых $K$ столбцах, тогда:
  \begin{equation*}
    |u_1| + |u_2| + \ldots + |u_S| \leqslant \sum_{n=1}^{N}\sum_{k=1}^{K}|a_{nk}| \leqslant A^* = \sum_{n,k=1}^{\infty}|a_{nk}| \implies
  \end{equation*}
  $\implies$ последовательность $U_i^* \nearrow$ и ограничена $\implies$ ряд $\sum_{r=1}^{\infty}u_r$ сходится абсолютно $\implies$ сходится.

  $"\impliedby"$ Пусть ряд $\sum_{r=1}^{\infty}|u_r|$ сходится $\implies \forall N,K \ \exists S:$ все числа $a_{11},a_{12},\ldots,a_{1K},a_{21},\ldots,a_{2K},\ldots,a_{N1},\ldots,a_{NK}$ содержатся среди чисел $u_1,\ldots,u_S$. Тогда
  \begin{equation*}
    A_{NK}^* = \sum_{n=1}^{N}\sum_{k=1}^{K}|a_{nk}| \leqslant\sum_{r=1}^{S}|u_r| \leqslant U^* = \sum_{r=1}^{\infty}|u_r| \implies
  \end{equation*}
  $\implies$ ряд $\sum_{n,k=1}^{\infty}a_{nk}$ сходится.

  Покажем, что $\sum_{n,k=1}^{\infty}a_{nk} = \sum_{r=1}^{\infty}u_r$.

  Так как ряд $\sum_{r=1}^{\infty}u_r$ сходится абсолютно, то расположим элементы по квадратам:
  \begin{equation*}
    \begin{array}{l}
      a_{11} = u_{r_1}                                       \\
      a_{12} + a_{22} + a_{21} = u_{r_2} + u_{r_3} + u_{r_4} \\
      \vdots                                                 \\
      A_{nn} = a_{11} + \ldots + a_{nn} = U_n = u_{r_1} + \ldots + u_{r_n}
    \end{array}
  \end{equation*}
  \begin{equation*}
    A = \underset{n\rightarrow\infty}{\lim}A_{nn} = \underset{n\rightarrow\infty}{\lim}U_n = U
  \end{equation*}
\end{proof}

\begin{proof}[Доказательство "Главной"]
  Из четырех предыдущих теорем $\implies$ "Главная"  теорема.
\end{proof}

\section{Поточечная и равномерная сходимость семейства функций}

\begin{definition}
  \textbf{Семейство функций} -- это произвольное множество функций.

  Пусть $f:X\times T \rightarrow Y$. Если по каким-либо соображениям элементам множества $T$ уделяется особое внимание, то будем их называть \textbf{параметрами}.

  То есть $\forall t \in T$ можно рассмотреть функцию
  \begin{equation*}
    f_t(x) = f(x,t).
  \end{equation*}

  В этом случае будем говорить, что задано семейство функций, зависящих от параметра $t$.
\end{definition}

\begin{example}
  $T = \mathbb{N}$, тогда $f_n(x) = x^n$.
\end{example}

Пусть задано семейство отображений $f_t: \ X\rightarrow Y_\rho, \ Y$ -- метрическое пространство с заданной метрикой $\rho, \ t \in T$.

Пусть $\mathfrak{B}$ -- база на $T$.

\begin{definition}
  Будем говорить, что семейство $\{f_t\}$ сходится в точке $x \in X$, если $f_t(x)$ как функция аргумента $t$ имеет предел по базе $\mathfrak{B}$, то есть $\exists y_x \in Y_\rho: \ \forall \epsilon > 0 \ \exists B \in \mathfrak{B}: \ \forall t \in B$
  \begin{equation*}
    \rho(f_t(x),y_x) < \epsilon.
  \end{equation*}
\end{definition}

\begin{definition}
  Множество $E = \{x \in X : \ \{f_t\}$ сходится в точке $x\}$ называется \textbf{областью сходимости} семейства $\{f_t\}$ по базе $\mathfrak{B}$.

  Далее, на $E$ введем функцию, положив
  \begin{equation*}
    f(x) = \underset{\mathfrak{B}}{\lim}f_t(x).
  \end{equation*}

  Функция $f(x)$ называется \textbf{предельной}.
\end{definition}

\begin{definition}
  Пусть дано семейство $f_t: X \rightarrow Y_u$ и $f: X \rightarrow Y$. Будем говорить, что $f_t$ сходится по базе $\mathfrak{B}$ \textbf{поточечно} к $f$ на $X$, если $\forall x \in X \ \forall \epsilon > 0 \ \exists B_x \in \mathfrak{B}: \ \forall t \in B_x$
  \begin{equation*}
    \rho(f_t(x),f(x)) < \epsilon.
  \end{equation*}

  Обозначение:
  \begin{equation*}
    f_t \underset{\mathfrak{B}}{\rightarrow} f \ (on \ X)
  \end{equation*}
\end{definition}

\begin{definition}
  Семейство $\{f_t\}$ сходится \textbf{равномерно} по базе $\mathfrak{B}$ к $f$ на $X$, если $\forall \epsilon > 0 \ \exists B \in \mathfrak{B}: \ \forall t \in B$ и $\forall x \in X$
  \begin{equation*}
    \rho(f_t(x),f(x)) < \epsilon
  \end{equation*}

  Обозначение:
  \begin{equation*}
    f_t \underset{\mathfrak{B}}{\rightrightarrows} f \ (on \ X)
  \end{equation*}
\end{definition}

\begin{definition}
  Пусть $f_n: X \rightarrow \mathbb{R}$ -- последовательность функций и $f: X \rightarrow \mathbb{R}$. Семейство $\{f_n\}$ \textbf{сходится поточечно} к $f$ на $X$, если $\forall x \in X \ \exists f(x) = \underset{n\rightarrow\infty}{\lim}f_n(x), \ \forall \epsilon > 0 \ \exists N: \ \forall n > N$
  \begin{equation*}
    |f_n(x) - f(x)| < \epsilon.
  \end{equation*}

  Обозначение:
  \begin{equation*}
    f_n \underset{n\rightarrow\infty}{\rightarrow}f \ (on \ X)
  \end{equation*}
\end{definition}

\begin{definition}
  Последовательность $\{f_n\}$ \textbf{равномерно сходится} к $f$ на $X$ при $n\rightarrow\infty$, если $\forall \epsilon > 0 \ \exists N \in \mathbb{N}: \ \forall n > N \ \forall x \in X$
  \begin{equation*}
    |f_n(x) - f(x)| < \epsilon.
  \end{equation*}

  Обозначение:
  \begin{equation*}
    f_n \underset{n\rightarrow\infty}{\rightrightarrows} f \ (on \ X)
  \end{equation*}
\end{definition}

\begin{example}
  $f_n: \mathbb{R}\rightarrow\mathbb{R}, \quad f_n(x) = x^n$

  Имеем при фиксирвоанном $x$:
  \begin{equation*}
    \underset{n\rightarrow\infty}{\lim}f_n(x) = \left\{\begin{array}{ll}
      0,        & -1 < x < 1     \\
      1,        & x = 1          \\
      + \infty, & x > 1          \\
      \nexists, & x \leqslant -1
    \end{array}\right.
  \end{equation*}

  Таким образом область сходимости этой последовательности $E = (-1;1]$. На множестве $E$ определим предельую функцию
  \begin{equation*}
    f(x) = \underset{n\rightarrow\infty}{\lim}f_n(x) = \underset{n\rightarrow\infty}{\lim}x^n = \left\{\begin{array}{ll}
      0, & x \in (-1;1) \\
      1, & x = 1
    \end{array}\right.
  \end{equation*}

  Покажем, что $f_n$ сходится к $f$ на $E$ неравномерно, то есть $\exists \epsilon > 0 \ \forall N \in \mathbb{N}: \ \exists n > N \ \exists x \in X:$
  \begin{equation*}
    |f_n(x) - f(x)| \geqslant \epsilon.
  \end{equation*}

  Возьмем $\epsilon = \frac{1}{2}$. Пусть $N$ задано произвольно. Возьмем $n = N + 1$ и $x: \ x^n = \frac{3}{4}$, то есть $x = \sqrt[n]{\frac{3}{4}}$. Тогда:
  \begin{equation*}
    |f_n(x) - f(x)| = \bigg|\bigg(\sqrt[n]{\frac{3}{4}}\bigg)^n - 0\bigg| = \frac{3}{4} > \frac{1}{2}
  \end{equation*}
\end{example}

\begin{example}
  $f_n(x) = \frac{x}{1 + n^2x^2}$

  $\forall x \in X$:
  \begin{equation*}
    \underset{n\rightarrow\infty}{\lim}f_n(x) = \underset{n\rightarrow\infty}{\frac{x}{1 + n^2x^2}} = 0.
  \end{equation*}

  Таким образом, $f(x) = 0 \ \forall x \in \mathbb{R}$. Покажем, что $f_n \underset{n\rightarrow\infty}{\rightrightarrows}f$ на $\mathbb{R}$.

  Имеем:
  \begin{multline*}
    |f_n(x) - 0| = \bigg|\frac{x}{1 + n^2x^2}\bigg| = \frac{1}{2n} \cdot \bigg|\frac{2nx}{1 + n^2x^2}\bigg| \leqslant \\
    \leqslant\left|\begin{array}{l}
      0 \leqslant (1 - nx)^2 = 1 + n^2x^2 - 2nx \implies \\
      \implies 2nx \leqslant 1 + n^2x^2
    \end{array}\right| \leqslant \\
    \leqslant \frac{1}{2n}\cdot 1 = \frac{1}{2n}
  \end{multline*}

  Пусть $\epsilon > 0$ задано. Возьмем $N: \ \forall n > N \ \frac{1}{2n} < \epsilon, \ N = [\frac{1}{2\epsilon}]$. Таким образом, $\forall n > N \ \forall x \in \mathbb{R}$
  \begin{equation*}
    |f_n(x) \leqslant\frac{1}{2n} < \frac{1}{2N} = \epsilon \implies
  \end{equation*}
  $\implies f_n(x) \underset{n\rightarrow\infty}{\rightrightarrows} f(x)$ на $\mathbb{R}^\infty$
\end{example}

\begin{example}
  $f_n(x) = \frac{n\cdot x}{1 + n^2 x^2}, \quad \forall x \in \mathbb{R} \quad \underset{n\rightarrow\infty}{f_n(x)} = 0$

  Покажем, что данное семейство не имеет равномерной сходимости к $f$.

  Рассмотрим $f_n(x) - f(x) = f_n(x) = \frac{n\cdot x}{1 + n^2 x^2}$:
  \begin{equation*}
    f_n'(x) = \frac{n\cdot (1 + n^2 x^2) - n \cdot x \cdot (2xn^2)}{(1 + n^2 x^2)^2} = \frac{n - n^3x^2}{(1 + n^2x^2)^2} = 0, \quad x = \pm\frac{1}{n}
  \end{equation*}

  Далее, $f_n(\frac{1}{n}) = \frac{1}{2}$. Возьмем $\epsilon = \frac{1}{4}$. Тогда если $N$ задано, то выберем $n = N + 1$ и $x = \frac{1}{n}$.

  Тогда $|f_n(x) - f(x)| \big|_{x = \frac{1}{n}} = \frac{1}{2} > \frac{1}{4} \implies f_n(x) \underset{n\rightarrow\infty}{\nrightrightarrows}f(x)$.
\end{example}

\begin{theorem}[критерий Коши сходимости семейства функций]
  Пусть $Y$ -- полное метрическое пространство (М.П.), $f_t:X \rightarrow Y, \ t \in T$ -- семейство $\{f_t\}$ равномерно сходится на $X$ по базе $\mathfrak{B} \iff \forall \epsilon > 0 \exists B \in \mathfrak{B}: \ \forall t_1,t_2 \in B$ и $\forall x \in X$
  \begin{equation*}
    \rho(f_{t_1}(x);f_{t_2}(x)) < \epsilon
  \end{equation*}
\end{theorem}

\begin{definition}[равномерная сходимость семейства функций по базе]
  Будем говорить, что семейство функций $f_t: X \rightarrow Y$ \textbf{равномерно сходится на $X$ по базе} $\mathfrak{B}$, если:
  \begin{enumerate}
    \item $\exists f: X \rightarrow Y:$
          \begin{equation*}
            \underset{\mathfrak{B}}{\lim}f_t(x) = f(x), \quad \forall x \in X
          \end{equation*}
    \item $f_t$ сходится равномерно к $f$ на $X$ по базе $\mathfrak{B}$.
  \end{enumerate}
\end{definition}

\begin{theorem}[формулировка критерия Коши для последовательности $f_n(x)$]
  Последовательность $f_n(x)$ равномерно сходится на $X \iff \forall \epsilon > 0 \ \exists N \in \mathbb{N}: \ \forall n > N \ \forall p > 0 \ \forall x \in X$
  \begin{equation*}
    \big|f_n(x) - f_{n+p}(x)\big| < \epsilon
  \end{equation*}
\end{theorem}

\begin{proof}
  $"\implies"$ Проведем доказательство для $Y = \mathbb{R}$.

  Пусть семейство $f_t$ сходится равномерно на $X$ по базе $\mathfrak{B}$, то есть $\exists f(x): X \rightarrow \mathbb{R}:$
  \begin{equation*}
    f_t(x) \underset{\mathfrak{B}}{\rightrightarrows}f(x).
  \end{equation*}

  Покажем, что выполнено условие Коши.

  Пусть $\epsilon > 0$ задано. Выберем $B \in \mathfrak{B}: \forall t \in B \ \forall x \in X$
  \begin{equation*}
    \big|f_t(x) - f(x)\big| < \frac{\epsilon}{2}.
  \end{equation*}

  Тогда $\forall t_1, t_2 \in B \ \forall x \in X$.
  \begin{multline*}
    \big|f_{t_1}(x) - f_{t_2}(x)\big| = \big|f_{t_1}(x) - f(x) + f(x) - f_{t_2}(x)\big| \leqslant \\
    \leqslant \big|f_{t_1}(x) - f(x)\big| + \big|f_{t_2}(x) - f(x)\big| < \frac{\epsilon}{2} + \frac{\epsilon}{2} = \epsilon.
  \end{multline*}

  $"\impliedby"$ Пусть $\forall \epsilon > 0 \ \exists B \in \mathfrak{B}: \forall t_1,t_2 \in B$ и $\forall x \in X$
  \begin{equation*}
    \big|f_{t_1}(x) - f_{t_2}(x)\big| < \epsilon \quad (\star)
  \end{equation*}

  Зафиксируем $x \in X$. Тогда выражение $(\star)$ есть точная формулировка критерия Коши существования предела функции $f_t(x)$ по базе $\mathfrak{B} \implies \forall x \in X \ \exists \underset{\mathfrak{B}}{\lim}f_t(x) = f(x)$.

  Покажем, что $f_t(x) \underset{\mathfrak{B}}{\rightrightarrows}f(x)$ на $X$.

  В $(\star)$ передем к пределу по базе $\mathfrak{B}$ по переменной $t_1$. Получим, что
  \begin{equation*}
    \big|f(x) - f_{t_2}(x)\big| < \epsilon.
  \end{equation*}

  Таким образом получаем равномерную сходимость семейства $f_{t_2}(x)$ к $f$ на $X$ по базе $\mathfrak{B}$, то есть $\forall \epsilon > 0 \ \exists B \in \mathfrak{B} \ \forall t_2 \in B$ и $\forall x \in X$
  \begin{equation*}
    \big|f_{t_2}(x) - f(x)\big| < \epsilon
  \end{equation*}
\end{proof}

\begin{effect}
  Пусть $X,Y$ -- метрические пространства, $E \subset X, \ x_0 \in E$ -- предельная точка для $E$. Семейство $f_t: X \rightarrow Y$:
  \begin{enumerate}
    \item $f_t$ сходится на $E$ по базе $\mathfrak{B}$;
    \item $f_t$ расходится в точке $x_0$ по базе $\mathfrak{B}$;
    \item $\forall t \ f_t$ непрерывно в точке $x_0$.
  \end{enumerate}

  Тогда на $E$ семейство $f_t$ сходится неравномерно.
\end{effect}

\begin{proof}
  Применим критерий Коши, покажем, что $\exists \epsilon > 0: \ \forall B \in \mathfrak{B} \ \exists t_1, t_2 \in B$ и $\exists x \in E$:
  \begin{equation*}
    \rho_Y(f_{t_1}(x), f_{t_2}(x)) \geqslant \epsilon.
  \end{equation*}

  Таким образом $f_t$ расходится в точке $x_0$, то $\exists \epsilon > 0: \forall B \in \mathfrak{B} \ \exists t_1,t_2 \in B$:
  \begin{equation*}
    \rho_Y(f_{t_1}(x_0), f_{t_2}(x_0)) \geqslant \epsilon.
  \end{equation*}

  Так как $f_{t_1}$ и $f_{t_2}$ непрерывны, то $\exists U(x_0)\subset X: \ \forall x \in U(x_0)$
  \begin{equation*}
    \rho_Y(f_{t_1}(x), f_{t_2}(x)) \geqslant \epsilon.
  \end{equation*}

  Возьмем $\forall x \in U(x_0) \cap E \implies$ тогда в $x$ будет выполняться неравенство
  \begin{equation*}
    \rho_Y(f_{t_1}(x), f_{t_2}(x))\geqslant E \implies
  \end{equation*}
  $\implies f_t$ на $E$ сходится неравномерно.
\end{proof}

\begin{effect}[из следствия]
  Если $f_t:(a,b] \rightarrow D, \ D$ -- область в $Y$:
  \begin{enumerate}
    \item $\forall t \ f_t$ непрерывно в точке $b$;
    \item $f_t$ сходится на $(a,b)$ по $\mathfrak{B}$;
    \item $f_t$ расходится в точке $b$.
  \end{enumerate}

  Тогда на $(a,b) \ f_t$ сходится неравномерно.
\end{effect}

\section{равномерная сходимость функциональных рядов}

\begin{definition}[функциональный ряд]
  Пусть $f_n: X \rightarrow\mathbb{R}, \ X$ -- произвольное множество.

  \textbf{Функциональным рядом} называется выражение вида
  \begin{equation*}
    (\triangle)\quad \sum_{n=1}^{\infty}f_n(x).
  \end{equation*}

  Говорят, что ряд $(\triangle)$ сходится на $X$ поточечно, если на $X$ сходится поточечно последовательность его частичных сумм. Ряд $(\triangle)$ равномерно сходится на $X$, если на $X$ равномерно сходится последовательность его частичных сумм.
\end{definition}

\begin{theorem}[критерий Коши равномерной сходимости функциональных рядов]
  Ряд $(\triangle)$ равномерно сходится на $X \iff \forall \epsilon > 0 \ \exists N: \ \forall n > N \ \forall p > 0 \ \forall x \in X$
  \begin{equation*}
    \big|f_{n+1}(x) + \ldots + f_{n+p}(x)\big| < \epsilon
  \end{equation*}
\end{theorem}

\begin{proof}
  Самостоятельно.
\end{proof}

\begin{effect}
  Если:
  \begin{enumerate}
    \item Ряд $(\triangle)$ сходится на $(a,b)$;
    \item Расходится в точке $b$;
    \item $\forall n \ f_n(x)$ непрерывно в точке $b$.
  \end{enumerate}

  Тогда ряд $(\triangle)$ сходится на $(a,b)$ неравномерно.
\end{effect}

\begin{proof}
  Следует из предыдущих следствий.
\end{proof}

\begin{definition}
  Ряд $(\triangle)$ сходится абсолютно на $X$, если на $X$ сходится ряд
  \begin{equation*}
    \sum_{n=1}^{\infty}\big|f_n(x)\big|
  \end{equation*}
\end{definition}

\begin{theorem}
  Пусть ряды
  \begin{equation*}
    (A) \ \sum_{n=1}^{\infty}a_n(x), \quad (B) \ \sum_{n=1}^{\infty}b_n(x)
  \end{equation*}
  такие, что:
  \begin{enumerate}
    \item $\forall n$ функции $a_n(x)$ и $b_n(x)$ определены на $X$;
    \item $\exists N: \ \forall n > N$
          \begin{equation*}
            \big|a_n(x)\big| \leqslant b_n(x) \quad \forall x \in X
          \end{equation*}
    \item Ряд $(B)$ сходится на $X$ равномерно.
  \end{enumerate}

  Тогда ряд $(A)$ сходится на $X$ равномерно.
\end{theorem}

\begin{proof}
  Пусть $\epsilon > 0$ задано. Выберем $N: \ \forall n > N, \ \forall p > 0 \ \forall x \in X$
  \begin{equation*}
    b_{n+1}(x) + \ldots + b_{n+p}(x) < \epsilon.
  \end{equation*}

  Тогда $\forall n > N, \ \forall p > 0, \ \forall x \in X$
  \begin{multline*}
    \big|a_{n+1}(x) + \ldots + a_{n+p}(x)\big| \leqslant \\
    \leqslant \big|a_{n+1}(x)\big| + \ldots + \big|a_{n+p}(x)\big| \leqslant b_{n+1}(x) + \ldots + b_{n+p}(x) < \epsilon \implies
  \end{multline*}
  $\implies$ по критерию Коши ряд $(A)$ сходится равномерно на $X$.
\end{proof}

\begin{effect}[Мажорантный признак Вейерштрасса]
  Пусть
  \begin{enumerate}
    \item $\forall n \exists M_n:$
    \begin{equation*}
      \big|a_n(x)\big| \leqslant M_n \quad \forall x \in X
    \end{equation*}
    \item Ряд $\sum_{n=1}^{\infty} M_n$ сходится.
  \end{enumerate}

  Тогда ряд $\sum_{n=1}^{\infty}M_n$ сходится на $X$ абсолютно и равномерно.
\end{effect}

\begin{definition}[неубывающая (невозрастающая) последовательность]
  Последовательность $f_n: X \rightarrow\mathbb{R}$ называется \textbf{неубывающей (невозрастающей)} на $X$, если $\forall x \in X$ последовательность $f_n$ не убывает (не возрастает).
\end{definition}

\begin{theorem}[признаки Абеля и Дирихле]
  \begin{enumerate}
    \item (Абеля)
    
    Пусть функции $a_n(x)$ и $b_n(x)$ удовлетворяют условиям:
    \begin{enumerate}
      \item Ряд $\sum_{n=1}^{\infty}a_n(x)$ сходится равномерно на $X$;
      \item Последовательность $\{b_n(x)\}$ равномерно ограничена на $X$ и монотонна (то есть $\exists L > 0: \ \forall n \in \mathbb{N}$ и $\forall x \in X \quad \big|b_n(x)\big| \leqslant L$).
    \end{enumerate}

    Тогда ряд
    \begin{equation*}
      \sum_{n=1}^{\infty}(a_n(x) \cdot b_n(x))
    \end{equation*}
    сходится на $X$ равномерно.

    \item (Дирихле)
    
    \begin{enumerate}
      \item Частичные суммы ряда $\sum_{n=1}^{\infty}a_n(x)$ равномерно ограничена на $X$ (то есть $\exists M > 0: \ \forall n$ и $\forall x \in X \quad |\sum_{k=1}^{n}a_k(x)| \leqslant M$)
    \end{enumerate}
  \end{enumerate}
\end{theorem}

\end{document}